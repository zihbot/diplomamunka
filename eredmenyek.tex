%----------------------------------------------------------------------------
\chapter{Eredmények}
%----------------------------------------------------------------------------

\section{Iris adathalmaz}
Először egy hagyományos adathalmazon, az \emph{iris}-en próbáltam ki az algoritmust, itt a tanulmányban publikáltakkal megegyező adatokat kaptam.

\section{Laboradatok}
\begin{table}[]
\begin{tabular}{lccc}
Paraméter & Referencia tartomány (mmol/L) & 1. típusú futtatás & 2. típusú futtatás \\
\hline
age  & 44.0 - 65.0 & 59.06                         & 41.29\\
HDLC & 0.9 - 1.5   & 1.395                         & 1.2, 1.605\\
TG  & 0.5 - 2.0   & 1.785, 3.54, 3.88             & 1.175\\
GLU   & 3.9 - 5.8   & 3.95, 4.75, 6.55, 6.7, $\dotsc$ & 5.55
\end{tabular}
\caption{Kapott eredmények}
\label{tab:eredmenyek}
\end{table}

\begin{figure}[htp]
    \centering
    \includegraphics[width=8cm]{figures/result-bayes.png}
    \caption{1. futtatás Bayes-hálója}
    \label{fig:result-bayes}
\end{figure}

A laboradatokkal az első futtatások (magas L érték) és második futtatások ($L = 3$) során kapott eredmények a \ref{tab:eredmenyek} ábrán látható. Az 1. futtatások során keletkezett Bayes-háló a \ref{fig:result-bayes} ábrán látható. A Második futtatásoknál egy láncba fűződtek a változók, úgyhogy a struktúra tanuláson mindenképpen finomítani kell (a pomeganate beépített \emph{Chow-Liu} struktúra tanulásához képest).

A triglicerideknél a kapott érték(ek) nem bíztatóak, jelentősen eltérnek a referencia tartománytól, valószínűleg egyszerűen két részre bontotta csak az intervallumot. A koleszterin és cukor felső határainak közelében viszont talált vágási pontot a 2. típusú futtatás. Valószínűleg itt az idős emberek száma nőtt meg, ezért sikerült megtalálni.

Számomra úgy tűnik, az algoritmus főleg egy "cutoff" pont meghatározására lehet jó, amely felett valamilyen változás van a többi változóban is.

\section{Folyatási lehetőségek}
A vizsgált folytonos paraméterek közelítették a normál eloszlást. Szerintem érdemes lenne megvizsgálni olyan paraméterekre, melyeknek az eloszlása ettől eltérő, és nem a referencia tartomány két határát, hanem egy "cutoff" pontot kell meghatározni.