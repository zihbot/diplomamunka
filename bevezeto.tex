%----------------------------------------------------------------------------
\chapter*{Bevezetés}\addcontentsline{toc}{chapter}{Bevezetés}
%----------------------------------------------------------------------------

Ez a dokumentum a Diplomatervezés 1. tárgyhoz készült. A Diplomamunka kiindulását tartalmazza, ám több részlet még nincs kifejtve benne valamint még az utolsó fejezetet nem tartalmazza, amelyben a valós adatokon történő kiértékelés lesz.

Kauzalitás modellezésére gyakori a Bayes-hálók használata. Ezen modellek diszkrét változók közötti kapcsolatot írják le. Ezért a tanításukat is diszkrét adatokon kell végezni. A mérhető, nyers adatok azonban gyakran tartalmaznak folytonos változókat. Ezeknek a diszkretizálására újabb eljárások kifejlesztése továbbra is folyamatban van. Jelen Diplomamunka az egyik új diszkretizációs eljárás köré épül és egy automatizált Bayes-háló tanító könyvtárat mutat be, mely ezt az eljárást használja.

A Bevezető fejezet tartalmazza a Diplomamunka célját, szerkezetét és a fő fejezetek tartalmát. Ez segít a dokumentum átlátásában.

A Fogalommagyarázat című fejezetben szerepelnek a statisztikai alapfogalmak. Később csak hivatkozás történik ezekre, ismeretük szükséges az algoritmus megértéséhez.

A Valószínűségelmélet fejezet összefoglalja a Bayes-i valószínűségelmélet azon részeit, melyeket a Diplomamuka később használ. Emellett példákat ad a könnyebb befogadáshoz. Itt szerepel az algoritmus megértéséhez szükséges Bayes-i következtetés ismertetése, valamint a Bayes-hálókhoz adott áttekintés.

A Diszkretizáció statisztikai módszerei fejezetben az erre használható algoritmusok leírása található. Áttekintést ad a diszkretizáció nehézségeiről, valamint az egyváltozós és többváltozós diszkretizáció különbségéről is.

Az Algoritmikus többváltozós diszkretizáció fejezetben a használt diszkretizációs eljárás logikai váza szerepel, a hozzá tartozó matematikai leírásokkal és értelmezésekkel együtt.

A Bayes-döntés alapján diszkretizáló modul fejezet dokumentálja az elkészült modult. A modul tartalmazza az előző fejezetben leírt algoritmus implementációját is, az algoritmus szerint végzi a diszkretizálást.

A lebegőpontos számok tizedesvessző helyett ponttal vannak jelölve. Ennek oka, hogy a program kimenete amerikai írásmód szerint formáz, és az egységesség érdekében máshol is így használom.