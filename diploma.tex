\documentclass[11pt,a4paper,oneside]{report}             % Single-side
%\documentclass[11pt,a4paper,twoside,openright]{report}  % Duplex

\PassOptionsToPackage{chapternumber=Huordinal}{magyar.ldf}
\usepackage[utf8]{inputenc}
\usepackage[T1]{fontenc}
\usepackage[unicode]{hyperref}
\def\magyarOptions{defaults=hu-min,suggestions=no}
\usepackage{amsmath}
\usepackage{amssymb}
\usepackage{enumerate}
\usepackage{enumitem}
\usepackage[thmmarks]{ntheorem}
\usepackage{graphics}
\usepackage{epsfig}
\usepackage{listings}
\usepackage{color}
%\usepackage{fancyhdr}
\usepackage{lastpage}
\usepackage{anysize}
\usepackage{footnote}
\usepackage[magyar]{babel}
\usepackage{csquotes}
\usepackage{sectsty}
\usepackage{setspace}  % Ettől a táblázatok, abrak, lábjegyzetek maradnak 1-es sorközzel!
\usepackage[hang]{caption}
\usepackage{listings}
\usepackage{xcolor}
\usepackage{blindtext}
\usepackage{scrextend}
\usepackage{tikz}
\usepackage{aeguill}
\usepackage{microtype}
\usepackage{interval}
\addtokomafont{labelinglabel}{\sffamily}
\usepackage[
backend=biber,
style=alphabetic,
sorting=nyt
]{biblatex}
\addbibresource{mybib.bib}

%--------------------------------------------------------------------------------------
% Main variables
%--------------------------------------------------------------------------------------
\newcommand{\vikszerzo}{Zih Botond}
\newcommand{\vikkonzulens}{Dr.~Antal Péter}
\newcommand{\vikcim}{Bayes-hálók tanulása paraméterek diszkretizálásának integrálásával}
\newcommand{\viktanszek}{Méréstechnika és Információs Rendszerek Tanszék}
\newcommand{\vikdoktipus}{Diplomatervezés 1.}
\newcommand{\argmin}{\mathop{\mathrm{arg\,min}}}
\newcommand\descitem[1]{\item{\bfseries #1}}

%--------------------------------------------------------------------------------------
% Page layout setup
%--------------------------------------------------------------------------------------
% we need to redefine the page style plain
% another possibility is to use the body of this command without \fancypagestyle
% and use \pagestyle{fancy} but in that case the special pages
% (like the ToC, the References, and the Chapter pages)remain in plane style

\pagestyle{plain}
%\setlength{\parindent}{0pt} % áttekinthetőbb, angol nyelvű dokumentumokban jellemző
%\setlength{\parskip}{8pt plus 3pt minus 3pt} % áttekinthetőbb, angol nyelvű dokumentumokban jellemző
\setlength{\parindent}{12pt} % magyar nyelvű dokumentumokban jellemző
\setlength{\parskip}{0pt}    % magyar nyelvű dokumentumokban jellemző

\marginsize{35mm}{25mm}{15mm}{15mm} % any size package
\setcounter{secnumdepth}{0}
\sectionfont{\large\upshape\bfseries}
\setcounter{secnumdepth}{2}
\singlespacing
\frenchspacing

%--------------------------------------------------------------------------------------
%	Setup hyperref package
%--------------------------------------------------------------------------------------
\hypersetup{
    %bookmarks=true,            % show bookmarks bar?
    %unicode=false,             % non-Latin characters in Acrobat’s bookmarks
    pdftitle={\vikcim},        % title
    pdfauthor={\vikszerzo},    % author
    pdfsubject={\vikdoktipus}, % subject of the document
    pdfcreator={\vikszerzo},   % creator of the document
    pdfproducer={Producer},    % producer of the document
    pdfkeywords={keywords},    % list of keywords
    pdfnewwindow=true,         % links in new window
    colorlinks=true,           % false: boxed links; true: colored links
    linkcolor=black,           % color of internal links
    citecolor=black,           % color of links to bibliography
    filecolor=black,           % color of file links
    urlcolor=black             % color of external links
}

%--------------------------------------------------------------------------------------
% Set up listings
%--------------------------------------------------------------------------------------
\lstset{
	basicstyle=\scriptsize\ttfamily, % print whole listing small
	keywordstyle=\color{black}\bfseries\underbar, % underlined bold black keywords
	identifierstyle=, 					% nothing happens
	commentstyle=\color{white}, % white comments
	stringstyle=\scriptsize\sffamily, 			% typewriter type for strings
	showstringspaces=false,     % no special string spaces
	aboveskip=3pt,
	belowskip=3pt,
	columns=fixed,
	backgroundcolor=\color{lightgray},
}
\def\lstlistingname{lista}

%--------------------------------------------------------------------------------------
%	Some new commands and declarations
%--------------------------------------------------------------------------------------
\newcommand{\code}[1]{{\upshape\ttfamily\scriptsize\indent #1}}

% define references
\newcommand{\figref}[1]{\ref{fig:#1}.}
\renewcommand{\eqref}[1]{(\ref{eq:#1})}
\newcommand{\listref}[1]{\ref{listing:#1}.}
\newcommand{\sectref}[1]{\ref{sect:#1}}
\newcommand{\tabref}[1]{\ref{tab:#1}.}

\DeclareMathOperator*{\argmax}{arg\,max}
%\DeclareMathOperator*[1]{\floor}{arg\,max}
\DeclareMathOperator{\sign}{sgn}
\DeclareMathOperator{\rot}{rot}
\definecolor{lightgray}{rgb}{0.95,0.95,0.95}

\author{\vikszerzo}
\title{\viktitle}
%--------------------------------------------------------------------------------------
%	Setup captions
%--------------------------------------------------------------------------------------
\captionsetup[figure]{
%labelsep=none,
%font={footnotesize,it},
%justification=justified,
width=.75\textwidth,
aboveskip=10pt}

\renewcommand{\captionlabelfont}{\small\bf}
\renewcommand{\captionfont}{\footnotesize\it}

%--------------------------------------------------------------------------------------
% Table of contents and the main text
%--------------------------------------------------------------------------------------
\begin{document}
\singlespacing

\pagenumbering{arabic}
\onehalfspacing
\include{titlepage}
\tableofcontents\vfill
%----------------------------------------------------------------------------
\chapter*{Bevezetés}\addcontentsline{toc}{chapter}{Bevezetés}
%----------------------------------------------------------------------------

Ez a dokumentum a Diplomatervezés 1. tárgyhoz készült. A Diplomamunka kiindulását tartalmazza, ám több részlet még nincs kifejtve benne valamint még az utolsó fejezetet nem tartalmazza, amelyben a valós adatokon történő kiértékelés lesz.

Kauzalitás modellezésére gyakori a Bayes-hálók használata. Ezen modellek diszkrét változók közötti kapcsolatot írják le. Ezért a tanításukat is diszkrét adatokon kell végezni. A mérhető, nyers adatok azonban gyakran tartalmaznak folytonos változókat. Ezeknek a diszkretizálására újabb eljárások kifejlesztése továbbra is folyamatban van. Jelen Diplomamunka az egyik új diszkretizációs eljárás köré épül és egy automatizált Bayes-háló tanító könyvtárat mutat be, mely ezt az eljárást használja.

A Bevezető fejezet tartalmazza a Diplomamunka célját, szerkezetét és a fő fejezetek tartalmát. Ez segít a dokumentum átlátásában.

A Fogalommagyarázat című fejezetben szerepelnek a statisztikai alapfogalmak. Később csak hivatkozás történik ezekre, ismeretük szükséges az algoritmus megértéséhez.

A Valószínűségelmélet fejezet összefoglalja a Bayes-i valószínűségelmélet azon részeit, melyeket a Diplomamuka később használ. Emellett példákat ad a könnyebb befogadáshoz. Itt szerepel az algoritmus megértéséhez szükséges Bayes-i következtetés ismertetése, valamint a Bayes-hálókhoz adott áttekintés.

A Diszkretizáció statisztikai módszerei fejezetben az erre használható algoritmusok leírása található. Áttekintést ad a diszkretizáció nehézségeiről, valamint az egyváltozós és többváltozós diszkretizáció különbségéről is.

Az Algoritmikus többváltozós diszkretizáció fejezetben a használt diszkretizációs eljárás logikai váza szerepel, a hozzá tartozó matematikai leírásokkal és értelmezésekkel együtt.

A Bayes-döntés alapján diszkretizáló modul fejezet dokumentálja az elkészült modult. A modul tartalmazza az előző fejezetben leírt algoritmus implementációját is, az algoritmus szerint végzi a diszkretizálást.

A lebegőpontos számok tizedesvessző helyett ponttal vannak jelölve. Ennek oka, hogy a program kimenete amerikai írásmód szerint formáz, és az egységesség érdekében máshol is így használom.
%----------------------------------------------------------------------------
\chapter{Fogalommagyarázat}
%----------------------------------------------------------------------------

Az alábbi fejezet a Diplomamunkában használt fogalmak jelentését, a megértéshez szükséges alapismereteket foglalja össze.

\section{Előfeldolgozás}
Egy statisztikai feldolgozásra kapott adatsoron nem alkalmazhatóak azonnal statisztikai módszerek. Az adatsorban különböző hibák előfordulhatnak. Ilyenek lehetnek:
\begin{itemize}
    \item Hiányok
    \item Nem megfelelő típusú értékek
    \item Inkozisztens adatok
\end{itemize}
Ezért a statisztikai vizsgálat előtt szükség van egy \emph{előfeldolgozás}i lépésre, amely során előáll az adatok megfelelő formája.

\section{Statisztikai változó, adatpont}
A statisztikai számításnál rendezett adatsorokat használunk. A rendezettség a sorokra és oszlopokra bontott adatokban jelenik meg. Ezért az adathalmaz felírható táblázatos formában. A táblázat oszlopai a statisztikai változók, vagy röviden \emph{változó}k. Ezeket a tulajdonságokat értékeljük ki a vizsgált entitásoknál. Az oszlopban szereplő értékek egyforma típusúak (szám, szöveg), értelmezésük azonos. Amennyiben egy fizikai mennyiséget ábrázol, akkor a mértékegysége minden értéknek azonos, vagy egyértelműen jelölt. Ha a mértékegységek különböznek, az előfeldolgozás során célszerű ugyanolyan mértékegységre váltani őket.

Az adatsor soraiban egy egyedhez tartozó értékek szerepelnek. Ezek jellemzik az adott egyedet az adatsor változói szerint. Az egyed tágabban is értelmezhető, nem szükséges hozzá a valódi világnak egy objektuma, de az egy sorban szereplő adatoknak kapcsolódniuk kell egymáshoz. Egy sort \emph{adatpont}nak is neveznek, mert a változók által kifeszített térben pontosan egy pontot jelöl ki.

A táblázatos forma miatt az adatsor nagy hasonlóságot mutat egy relációs adatbázis táblával vagy nézettel. A fogalmak is összepárosíthatóak. A statisztikai változó megfelel egy adatbázis attribútumnak, mezőnek. A statisztikai adatpont pedig értelmezhető egy adatbázis rekordként. Emiatt a kapcsolat miatt egy relációs adatbázisból kiolvasott adatokkal egyszerűen lehet statisztikai számításokat végezni.

\section{Folytonos és diszkrét változó}
A statisztikai változók legfontosabb csoportosítása \cite{statokosvaltozok}. \emph{Folytonos} változók közé tartoznak azok, melyeknek értéke bármilyen valós számot felvehet, esetleges intervallum megszorítások mellett. Vagyis a felvehető értékek száma végtelen. Az adatsorban természetesen véges mennyiségű adat fog szerepelni, tehát egy folytonos változó \emph{megfigyelt} értékei megszámlálhatóak, de végtelen érték értelmezhető lenne az adott változó értékeként. Vannak egymáshoz közeli, és egymástól távolabbi értékek, a távolságuk kivonással megállípítható. Például ilyen a pontos életkor tizedes törtként években megadva, vagy egy növény levelének a hossza.

Ezzel szemben a \emph{diszkrét} változók értékei végesek, pontosan meghatározható, hogy két adatpont egy diszrét változóban megegyezik vagy eltér. A diszkrét változóknak nincs távolságuk, páronként különbözőek. Ilyen például a biológiai nem, vagy egy növény fajtája.

\section{Regresszió és klasszifikáció}
Gépi tanulási modellek csoportosításának egy módja. Ezek a modellek egy statisztikai adatsorból állnak elő, a vizsgált csoportosítás pedig a kimenetük alapján történik. A modelleknek a kiértékelés során van egy célváltozója, amely értékek a kiértékelendő adatok között nem szerepelnek, és amely értékét a modell alapján szeretnénk meghatározni. Amennyiben ez a célváltozó folytonos, úgy \emph{regresszió}ról beszélünk, ha pedig diszkrét, akkor \emph{klasszifikáció}ról. Több gépi tanulásos modell esetében a klasszifikáció is egyfajta regresszió, mert a modell folytonos értékéket állít elő, azt diszkretizálva kaphatóak meg az osztályok.

\section{Osztály}
A diszkrét változók értékeit \emph{osztály}oknak nevezzük. Az osztályozás feladata meghatározni, hogy az adott diszkrét változó melyik értéket veszi fel. Az osztályozás emiatt megegyezik a klasszifikációval, a magyar szakirodalomban így használják, ám a regresszió párjaként klasszifikációként hivatkoznak rá.

\section{Kvantálás}
A fizikai mérőeszközök analóg módon tudják mérni a jelet. Az adattárolás napjainkban nagy többségében digitális alapú. Digitálás számábrázolásnál nem lehet végtelenül pontosan eltárolni az értékeket, el kell férjen az operációs rendszertől és programozási nyelvtől függő, 32-128 bájtos adatként. Az analóg-digitális konverzió során ezért az analóg eszköz által mért érték kerekítve lesz. Ez nem okoz gondot, hiszen az eszköznek mérési hibája is van, ami általában nagyobb a kerekítési hibánál. Ezt a kerekítési folyamatot nevezik \emph{kvantálás}nak. Ebből adódóan az adatsor folytonos értékei is kisebb számosságú almazból kerülnek ki, mint a valódi értékek, hiszen több érték is ugyanarra a kvantálási szintre lesz kerekítve. Ezért mért értékek esetében érdemes figyelembe venni, hogy még a folytonos változók értékei között is lehetnek megegyezőek.

\section{Diszkretizáció}
A diszkretizáció folyamata során egy folytonos változóból egy diszkrét változó lesz. A folyamat végrehajtása olyan algoritmussal történik, amely a folytonos változónak nem csak az adatsorban szereplő értékeihez tud osztályt rendeli, hanem minden lehetséges értékéhez. Tágabb értelemben azt az algoritmust is \emph{diszkretizáló algoritmus}nak nevezzük, amely előállít egy olyan függvényt, amely segítségével diszkretizáció végezhető. A Diplomamunka ebben az értelemben használja a fogalmat.

\section{Feltételes valószínűségi tábla}
Néhány diszkrét valószínűségi változó közötti összefüggés megadására alkalmas táblázat. Tartalmazza az egyik változó feltételes valószínűségét a többi változóra vonatkoztatva. A nagysága a bemeneti változók számával exponenciálisan változik.

%Kardinalitás
%diszkretizációs intervallum
%hisztogram
%----------------------------------------------------------------------------
\chapter{Valószínűségelmélet}
%----------------------------------------------------------------------------

Az alábbi fejezetben a Bayes-tétel és annak következményei szerepelnek. Ezek ismerete szükséges a Diplomaterv központi algoritmusának megértéséhez.

\section{Bayes-tétel}
Jelölje az A esemény bekövetkezésének valószínűségét $P(A)$, $AB = A \cap B$ azt az eseményt, amikor A és B is bekövetkezik, és $P(A|B)$ az A feltételes valószínűségét B-re nézve. Ha A és B független események a feltételes valószínűség definíciójából következik, hogy
$$P(AB) = P(A \cap B) = P(A|B) \cdot P(B) =  P(B|A) \cdot P(A) $$
Ebből egy osztással megkapható a tétel állítása
$$P(A|B) = \frac{P(B|A) \cdot P(A)}{P(B)}$$
amely segítségével egy feltételes valószínűség az események valószínűsége és az ellenkező irányú feltételes valószínűség ismeretében meghatározható. Tehát van a feltételes valószínűségek között egy megfordíthatóság \cite{laszlo2011bayesi}.

B esemény felbontható a $\{A_i\}$ teljes eseményrendszerre, vagyis a $P(A_i \cap A_j) = 0$, ha $i \neq j$ és $ P(\bigcup_{i} A_i) = 1$. Ez azt jelenti, hogy az eseményteret olyan eseményekre bontjuk, amelyek közül minden esetben pontosan egy következik be és az egyik közülük mindenképpen bekövetkezik. Ilyenkor is alkalmazható marad a \emph{Bayes-tétel}:
$$P(A_i|B) = \frac{P(B|A_i) \cdot P(A_i)}{\sum_{j} P(B|A_j) \cdot P(A_j)}$$
Eszerint ha egy teljes eseményrendszer valószínűségei és hozzá kapcsolódóan egy másik B esemény feltételes valószínűségei ismertek, akkor meghatározható az eseményrendszer eseményeinek a B eseményhez tartozó feltételes valószínűségei.

\section{Bayes-háló}
A teljes eseményrendszerek közötti kapcsolatok irányított gráf formában is ábrázolhatók. A gráfban a csomópontok a teljes eseményrendszerek az élek pedig a közöttük lévő közvetlen összefüggéseket reprezentálja. Ilyenkor az élek irányának nincs különösebb jelentősége. Az így felépített irányított körmentes gráf egy \emph{Bayes-háló}. A körmentesség abból adódik, hogy csak a közvetlen összefüggések vannak éllel jelölve, így nem tud bezárulni egy kör, mert a két változó közt nem lehet egyszerre áttételes és direkt függés is.

A reprezentáció értelmezése kiterjeszthető. Egyrészt az eseményrendszerek helyett a csomópontokba beilleszthetők valószínűségi változók is. Másrészt az élek menti kapcsolatokat súlyozhatjuk a két valószínűségi változó feltételes valószínűségeivel, amelyben a kiindulópontra vonatkozik a célpontnak a feltételes valószínűsége. Az irány emiatt fontossá válik, az él megfordításakor a Bayes-tétellel számolható ki az ellentétes irányú feltételes valószínűség.

Így a Bayes-háló alkalmassá válik tudásreprezentációra is. Ebben az értelmezésben az élek iránya ok-okozati összefüggést mutat. A háló felépítése szakértőre vagy egy szakértői rendszerre van bízva. Ilyenkor a szakértő megkötéseket tehet, milyen kapcsolatok kell teljesüljenek vagy nem teljesülhetnek, valamint a valószínűségekre és feltételes valószínűségekre is adhat becslést vagy rögzítheti azokat.

\begin{figure}[htp]
    \centering
    \includegraphics[width=10cm]{figures/app/pelda-bayes.png}
    \caption{Tudásreprezentáció Bayes-hálóval}
    \label{fig:bayes-halo-tudasrep}
\end{figure}

Emellett ismert adatsorok alapján is felépíthetők ilyen rendszerek. Jelen Diplomaterv célja egy olyan könyvtár készítése, mely egy statisztikai adatsor alapján építi fel a tudásreprezentációra szolgáló Bayes-hálót.

A \ref{fig:bayes-halo-tudasrep} ábrán látható, hogyan használható a Bayes-háló tudásreprezentációra. Egy számítógép processzorának meghibásodását, és annak jeleit modellezi. A valószínűségi változók ebben az esetben binárisok, a + jelöli az igaz értéket. A processzor meghibásodásának valószínűsége 1\%. Annak a valószínűsége, hogy nem hibásodott meg, értelemszerűen $100\%-1\%=99\%$. Látható, hogy a $P(-)$\footnote{Ez egy egyszerűsített jelölésmód. Az X valószínűségi változóra a $P(+)$ azt a valószínűséget jelzi, hogy X igaz értéket vesz fel, tehát $P(+) = P(X=+) = P(X=True)$. Amikor egyértelmű, mely változóra vonatkozik a jelölés, ebben a Diplomamunkában az első jelölésmódot alkalmazom.}, $P(-|+)$, $P(-|-)$ kifejezések felírása redundáns, mivel a $+$ párjaikat 1-re egészítik ki.

Szintén leolvashatók a feltételes valószínűségek értékei. Például annak a valószínűsége, hogy Zaj keletkezik, amennyiben a processzor hűtése nem hibás, azt a bal oldali táblázat $P(+|-)$ valószínűségeként találjuk meg, ami 35\%.

A \emph{Zaj} és \emph{Túlmelegedés} változók határeloszlása\footnote{A változó igaz és hamis értékének valószínűsége. Például a Zaj változóra P(+).} szintén redundáns információ, ezek nem is szerepelnek az ábrán. Ezek a Bayes-hálóban történő következtetéssel számolhatók ki. Annak a valószínűsége, hogy hallunk zajt, miközben a másik két változóról nem tudunk semmit, kiszámolható a CPU hűtés hiba és a feltételes valószínűségek szorzatai összegéből.
$$P(Z=+) = P(Z=+\,|\,C=+) \cdot P(C=+) + P(Z=+\,|\,C=-) \cdot P(C=-) =$$
$$= 0.7 \cdot 0.01 + 0.35 \cdot 0.99 = \textbf{0.3535} $$
A CPU hűtés hiba valószínűségi változót C, a Zaj-t Z jelöli. A nem hallható zaj valószínűségéhez innen kétféleképpen is eljuthatunk. A fenti képletet alkalmazzuk a nem hallható zajra vagy 100\%-ból kivonjuk a Zaj valószínűségét.
$$P(Z=-) = 0.3 \cdot 0.01 + 0.65 \cdot 0.99 = \textbf{0.6465} = 1-0.3535 $$

A Zaj-ból a processzor túlmelegedésére következtetés a Bayes-tétel alapján történhet.
$$P(C=+\,|\,Z=+) = \frac{P(Z=+\,|\,C=+) \cdot P(C=+)}{P(Z=+)} = \frac{0.7 \cdot 0.01}{0.3535} \approx \textbf{0.0198}$$
Ehhez hasonlóan a Bayes-hálóban bármely változó rögzítése esetén a többire valószínűség számolható.

% TODO: Háló elemei, elnevezések

\section{Becslés}
A valószínűségek segítségével becsléseket is adhatunk arra vonatkozóan, hogy a megfigyelések mellett mely állapot a legvalószínűbb. Ezek általánosan használhatók paraméterezhető modellekre.

\subsection{Legnagyobb valószínűség módszer}
Ez a módszer intuitívan adódik. Maximális valószínűséget keres az adott megfigyelt adatokhoz. Azt vizsgálja, hogy a lehetséges paraméterezések ($\Lambda$) közül melyik mellett a legnagyobb a megfigyelt adat ($D$) valószínűsége.
$$\argmax_{\Lambda}{P(D\,|\,\Lambda)}$$
A maximumkereséshez a valószínűség negatív logaritmusát lehet venni, így numerikusan stabilabb lesz, és ennek keresni a minimumát. Minimumkeresésre több módszert ismerünk, például a deriváltjának nullhelyét keressük. Ez a módszer jól alkalmazható, ha a modell felírható függvényként, és nem áll rendelkezésünkre előzetes ismeret a paraméterekről, csak a modell és a megfigyelések.

\subsection{Bayes-döntés}
Amennyiben van további ismeretünk a paraméterekről, úgy alkalmazhatunk olyan becslést, amely ezt figyelembe veszi. A Bayes-tétel segítségével vizsgálható a megfigyelt adathoz tartozó legvalószínűbb paraméterezés is, hiszen
$$\argmax_{\Lambda}{P(\Lambda\,|\,D)} = \argmax_{\Lambda}\{P(D\,|\,\Lambda) \cdot P(\Lambda)\}$$
Ennek a becslés típusnak a neve \emph{Bayes-döntés}, más néven \emph{a postreriori} vagy \emph{posterior becslés}. A negatív logaritmus minimalizálása itt is jó megoldás. A Legnagyobb valószínűség módszerben (maximum likelihood) szereplő tényezőt - $P(D\,|\,\Lambda)$ - \emph{likelihood} tagnak, az előzetes ismeretet tartalmazót - $P(\Lambda)$ - \emph{prior} tagnak nevezik. A tag elnevezés a gyakori negatív logaritmizálás miatt van, mert annak hatására a szorzatból összeg lesz, melynek ezek a tagjai.

%Valószínűségelmélet - Bayes-tétel, Bayes-döntés, Bayes-háló, A-priori, posterior becslés
%----------------------------------------------------------------------------
\chapter{Jelenlegi módszerei}
%----------------------------------------------------------------------------

Az előző fejezetben leírt Bayes-hálók csomópontjai folytonos és diszkrét valószínűségi változók is lehetnek. Az automatizált struktúra tanulás során azonban az élek feltételes valószínűségi táblákként adódnak, ezek az algoritmusok csak diszkrét valószínűségi változókkal tudnak dolgozni. Ezért szükséges, hogy a bemeneti folytonos változók a struktúra tanulás előtt diszkretizálva legyenek.

Ez a fejezet bemutatja a diszkretizáció jelenlegi statisztikai módszereit. Példákat mutat az egyszerűbbek használatára és működésére. Emellett a jelenlegi Bayes-háló struktúra tanulási módszereket is leírja.

\section{Egyváltozós diszkretizáció}
A diszkretizációhoz az egyszerűbb megoldás, hogy minden változó a többitől függetlenül kerül diszkretizálásra. Az adatsor változatlansága esetén a változó diszkretizációja konstans marad.

A diszkretizációs intervallumok számának meghatározására több lehetőség van. Meghatározható egy konstans. Ezt szakértőre kell bízni, vagy egy viszonylag kicsi - de legalább 2 - számot választani, amihez emberi beavatkozásra van szükség. Dinamikusan is meghatározható. Ilyenkor az eredeti adatsor diszkrét változói közül a legnagyobb kardinalitású kiválasztható. Ezt a számosság értéket választjuk az intervallumok számának.

\begin{table}[h]\centering
    \begin{tabular}{ccc}
    A & B    & C    \\ \hline
    2 & 4.28 & 0.51 \\
    1 & 2.58 & 0.95 \\
    0 & 3.37 & 0.35 \\
    0 & 3.06 & 0.32 \\
    1 & 3.67 & 1.59 \\
    2 & 3.70 & 1.27 \\
    2 & 2.78 & 1.32 \\
    1 & 4.23 & 0.52 \\
    2 & 4.80 & 0.71 \\
    2 & 1.90 & 1.88
    \end{tabular}
    \caption{Véletlenszerűen generált adatpontok diszkretizációs példák bemutatásához}
    \label{tab:diszkretizacio-pelda}
\end{table}

Tekintsük a \ref{tab:diszkretizacio-pelda}. táblázat adatait. Ebben az esetben az \emph{A} változó diszkrét a \emph{B} és a \emph{C} változók folytonosak. A diszkrét változó értékei $\left\{ 0, 1, 2 \right\}$, így a kardinalitása \emph{3}. Ezért a példákban három intervallumra lesznek a folytonos változók osztva.

Az intervallumok többféle módon is megadhatók. Szövegesen leírható például így: a \emph{C} változó adatsorát az 1-nél nagyobb és nemnagyobb értékekre bontjuk. Ez matematikai jelölésekkel felírható a $\Lambda_{C} = \left\{ \interval[open left]{-\inf}{1}, \interval[open]{1}{\inf} \right\} $ módon, így $\Lambda$ jelöli az adatsor összes diszkretizációs intervallumának halmazát, az alsó index mutatja, mely változóhoz tartozik. Mivel minden pontot valamelyik intervallum tartalmazni fogja, ezért egy listában növekvő sorrendben elég megadni a határokat. Ráadásul a minimum és maximum érték is ismert, így ezek használhatók a $\inf$ szimbólum helyett vagy el is hagyhatók. Ezen a módon a $\Lambda_{C} = \langle 0.32, 1, 1.88 \rangle$. Ez a diszkretizációs sorozat \cite{friedman1996discretizing}. Függvény formában is leírható, ilyenkor
$f_{\Lambda_{C}}(x) = \left\{ \begin{array}{ll}
    0 & \mbox{ha } x \leq 1 \\
    1 & \mbox{ga } x > 1
\end{array} \right.$. Növekvő sorrendbe rendezett értékek esetén az intervallum utolsó elemének pozíciója is egyértelműen megadja a határt. A példában 6 db 1-nél nemnagyobb érték van, így $e_{C} = \langle 6 \rangle$.

\subsection{Egyenlő hosszú intervallumok}
A vizsgált változó értékkészletének tartománya felosztható az elvárt kardinalitásnak megfelelő, egyenlő hosszúságú intervallumra. Ennek kiszámolásához csak egy maximum és minimum keresésre van szükség, így rendkívül gyors és egyszerű. A diszkretizált értékek ilyenkor az eredeti értékek hisztogramján mutatják, hogy az adott érték melyik vödörbe került. Kiugró értékek esetén előfordulhat, hogy az egyes intervallumokba nagyon különböző számú adatpont esik, így a sok adatpontot tartalmazó intervallumokban az adatpontok közötti eltérések könnyen elvesznek, nem tanulható meg a Bayes-háló által. Ezért ezt a típusú felosztást egyenletes eloszlást közelítő adatsorokon érdemes használni, hogy ne legyen kevés adatpontot tartalmazó intervallum.

A példa esetében a legnagyobb folytonos értékek $\max \mathcal{X}_{B} = 4.8, \max \mathcal{X}_{C} = 1.88$, a legkisebb értékek pedig $\min \mathcal{X}_{B} = 1.9, \min \mathcal{X}_{C} = 0.32$. Három intervallumra kell osztani az értékeket, ezt jelölje $L=3$. Az \emph{i.} határ egyszerűen megkapható ($1 \leq i < L$):
$$ \Lambda_{V, i} = \frac{\max \mathcal{X}_{V} - \min \mathcal{X}_{V}}{L} \cdot i + \min \mathcal{X}_{V} $$
Így a példában a határok $\Lambda_{B} = \langle 2.92, 3.69 \rangle$ és  $\Lambda_{C} = \langle 0.84, 1.36 \rangle$.

\subsection{Egyenlő mintaszámú intervallumok}
A minták növekvő sorrendjében minden $N / L$. pozícióban található érték számít az intervallum határának. Ha az így számolt pozíció nem lenne egész szám, a kapott értéket a kerekítés szabályai szerint egész számra kerekítve választjuk ki a pozíciót. Ezeken a helyeken szereplő értékek adják az intervallumok határát. Hisztogramról leolvasható, mert L felbontású hisztogramon az adatponthoz az azt tartalmazó vödör sorszáma rendelhető. Egyenlő kvantilisű vagy egyenlő frekvenciájú intervallumoknak is nevezhetők.

A példában 10 minta szerepel, ezeket kell 3 intervallumban elhelyezni. Ezért az adott változó szerint sorbarendezett mintákban a 3.33 és 6.67 pozícióban levő értéket keressük. Tehát a 3. valamint 7. helyen levő értékek lesznek az intervallumok felső határai. Esetünkben így ezek a határok $\Lambda_{B} = \langle 2.78, 3.70 \rangle$ és  $\Lambda_{C} = \langle 0.51, 1.27 \rangle$.

\subsection{Nyomaték párosítás}
A diszkretizált értékeket $\langle d_{1}, d_{2}, \dots, d_{L-1}\rangle$ formában keressük, melynek a diszkrét eloszlását a $\langle p_{1}, p_{2}, \dots, p_{L-1}\rangle$ valószínűségek adják meg, és teljesítik a következő feltételeket \cite{nojavan2017comparative}:
\begin{align}
    p_{1}d_{1} + p_{2}d_{2} + \dots + p_{L-1}d_{L-1} &= q x_{1} + q x_{2} + \dots + q x_{N} = m \\
    p_{1}(d_{1} - m)^{2} + \dots + p_{L-1}(d_{L-1} - m)^{2} &= q(x_{1} - m)^{2} + \dots + q(x_{N} - m)^{2} \\
    p_{1}(d_{1} - m)^{2L-1} + \dots + p_{L-1}(d_{L-1} - m)^{2L-1} &= q(x_{1} - m)^{2L-1} + \dots + q(x_{N} - m)^{2L-1} \\
    p_{1} + p_{2} + \dots + p_{L-1} &= 1 \\
    q &= \frac{1}{N}
\end{align}
ahol \emph{x} a folytonos változó értékei, \emph{N} a számosságuk. Az \emph{m} került bevezetésre az adatsor átlagának jelölésére, a \emph{q} pedig az adatsort normalizáló tényező.

A diszkretizáció elvégzése után az új eloszlás minden rendbeli nyomatéka a megegyezik az eredeti adatsor adott nyomatékával. A diszkrét értékek számának növelésével növekszik az egyenletek száma és komplexitása. Az egyenletrendszert numerikusan is nehéz megoldani, mert nem konvex probléma, így az összes lehetséges értéken kell keresést végezni. A lehetséges értékek pedig a $ \mathbb{R}^{L-1} $ halmazban vannak. Lehetséges csak egy bizonyos pontosságig végezni a keresést és a minimum és maximum közé szorítani, így véges számú értéket kell kipróbálni. Ám az intervallumok számával exponenciálisan növekszik a lehetséges értékek száma, így könnyen végrehajthatatlanná válik.

\subsection{Tárgyterület-függő diszkretizáció}
Amennyiben az adott statisztikai változóról több információ áll rendelkezésre, készíthető speciális, az adott tárgyterülethez kapcsolódó diszkretizáció.

Ilyen lehet a laboratóriumi paraméterek \cite{katayev2010establishing} diszkretizálása, amely a jelenlegi gyakorlatot ülteti át algoritmikus formába. A mostani, erre a célra gyűjtött adatok helyett nagy mennyiségű, más mérési céllal készült statisztikai adatokkal határozza meg a referencia tartományt.

\section{Többváltozós diszkretizáció}
Az egyváltozós diszkretizáció egyszerűségéből adódik a hátránya is. A Bayes-hálóra azért van szükség, mert a változók között vannak összefüggések. Amikor egy-egy változó diszkretizációja izoláltan történik, a tartalmazott információ mennyisége csökken, és ez torzítja az összefüggéseket. Ezért a függő változók diszkretizációja során érdemes lehet figyelembe venni a kapcsolatokat, így az azokból származó információ kevésbé veszik el.

\subsection{Minimális hosszúságú leírás elve alapján}
A minimális hosszúságú leírás elve (MDL) azt mondja ki, hogy egy adathalmaz legjobb modellje az, amelyik minimalizálja a leírásához szükséges információ mennyiségét \cite{friedman1996discretizing}.

Az információ sűrűségének jellemzésére megfelelő mutató az (információ) entrópia. Így a feladatot megfogalmazható úgy is, hogy minimalizálandó a diszkretizációs elv szerint a modell entrópiája. Az algoritmus célja egy olyan reprezentációt találni az eredeti \emph{D} adatsorhoz, amely leírása bitekben minimális hosszúságú és tartalmazza a diszkretizált értékeket (részben a Bayes-hálóba kódolva), a diszkretizációs elvet, valamint a diszkretizált értékekből az eredeti adatsor visszaállításához szükséges információt.
$$ DL_{D} = DL(U^{*}, G, D^{*}) + DL_{\Lambda}(\Lambda) + DL_{D^{*} \rightarrow D}(D, \Lambda)$$
Itt a \emph{DL} a leírás hossza bitekben, \emph{D} az eredeti adatsor, \emph{G} a Bayes-háló, \emph{D*} a diszkretizált adatsor, \emph{$\Lambda$} a diszkretizációs elv és \emph{U*} a \emph{D*} egyedi értékei.

Ennek megoldása $O(n^3)$ időben futtatható, ahol az $n$ a tanító adatok száma.

\subsection{Bayes-döntés alapján}
A diszkretizációs elv kiválasztható úgy, hogy mindegyikre kiszámoljuk \textit{a posteriori} becsléssel, mennyire valószínűek a megadott adatokon, és ezek közül kiválasztjuk a legnagyobbat. A Bayes-tétel miatt ez felírható a likelihood és a prior szorzataként.
$$ \argmax_{\Lambda}{P(\Lambda | D)} =
\argmax_{\Lambda}{P(D | \Lambda) \cdot P(\Lambda)} $$
ahol $\Lambda$ a diszkretizációs elv, $D$ az adat.

Az \textit{a posteriori} becslés esetén a prior megfelelő megválasztása kulcsfontosságú. A könnyebb programozhatóság kedvéért a valószínűség helyett annak negatív logaritmáltját érdemes minimalizálni. Mivel a logaritmus szigorúan monoton növekvő függvény, ez nem változtat az eredményen, de a lebegőpontos számábrázolás miatt pontosabban számolható számítógépen.

Mivel megoldható a feladat dinamikus programozással (lehet építeni a korábban kiszámolt valószínűségekre), így az futásidő $O(n^2)$-re csökken, ahol $n$ a tanító adatok száma.

Az algoritmust a \ref{chapter:bayesdontesalapu}. fejezet részletesebben tárgyalja.

\section{Struktúra tanulás}
Egy Bayes-háló alapja egy irányított körmentes gráf. A tartalmazott változók közti függőségeket ez reprezentálja. Ezért hasznos egy adatsor alapján a függőségi hálót felépítő algoritmus, mert erre építve a Bayes-háló átmenetei már könnyen kiszámíthatók.

Általánosan két megközelítésből kiindulva működnek ezek az algoritmusok \cite{tsamardinos2006max}.

\subsection{Keresés és pontozás}
Ezek az algoritmusok a lehetséges struktúrákon haladnak, mindegyiket pontozzák az adatsorhoz illeszkedésük szerint. A pontszámokat összehasonlítják és a legjobb eredmény elért lesz a kiválasztott háló.

Ilyenek például a a \emph{pomegranate} csomagban megvalósított \emph{greedy} és \emph{exact} algoritmusok \cite{schreiber2018pomegranate}. Ezek szintén a Minimális hosszúságú leírás elve alapján döntenek, ám itt a diszkretizáció adott, a Bayes-háló struktúra változik. Emellett a struktúrától független tagokat egy \emph{büntetés} együtthatóban egyesíti, amely megváltoztatható. Így már nem a minimális hosszúságú leírással rendelkező struktúra lesz a kiválasztott, lehet több és kevesebb élt is engedélyezni.

A különbség a két említtet algoritmus között, hogy a \emph{greedy} mohó algoritmus lévén egyesével mindig a legtöbbet javító élt adja hozzá. Míg az \emph{exact} algoritmus A* keresést végez az összes háló struktúra felett.



%Diszkretizáció statisztikai módszerei - Egyváltozós, többváltozós
%----------------------------------------------------------------------------
\chapter{Algoritmikus többváltozós diszkretizáció}
\label{chapter:bayesdontesalapu}
%----------------------------------------------------------------------------
A Diplomamunka a Bayes-döntés alapú diszkretizációs eljárást valósította meg. Ezért az alábbi fejezetben ennek részletesebb ismertetése történik. Az előző fejezetben láthattuk, hogy az algoritmus lényege a következő képlet minimalizálása:
$$ \argmin_{\Lambda}{(-\ln{P(D | \Lambda)} - \ln{P(\Lambda)})}$$
Az első tag a likelihood, a második a prior.

\section{Prior}
A tanulmányban \cite{chen2017learning} leírt algoritmus prior tagja az alábbi:
$$\sum_{i=1}^{k-1} -\ln{(1-\exp{(
-L \cdot \frac{x_{\lambda_i+1}-x_{\lambda_i}}{x_{n}-x_{1}})})} +
\sum_{i=1}^{k}
-L \cdot \frac{x_{\lambda_i}-x_{\lambda_{i-1}+1}}{x_{n}-x_{1}}$$
ahol az $x_i$ az $i$. értéke a folytonos adatsornak (növekvő sorrend esetén), $\lambda_i$ a vizsgált diszkretizációs elv $i$. tartomány végpontjának indexe, $k$ az intervallumok száma, $n$ az adatok száma és $L$ egy választható együttható, amelyhez közelíteni fog az intervallumok száma.

A tag értéke akkor lesz alacsony, amikor az intervallumok két végpontja közötti különbség a teljes adatsor hosszának $L$-ed része. Ezzel elérhető, hogy az eredmény intervallumokból nagyjából $L$ darab legyen, hiszen minél jobban eltérnének ettől a hosszaik, annál nagyobb büntetést kapna.

\section{Likelihood}
A likelihood tag az alábbi \cite{boulle2006modl}:
$$
\sum_{i = 1}^{k} \left[ \ln \binom{\gamma_i+J_P-1}{J_P-1} + \ln \frac{\gamma_i!}{n_{i,1}^{(P)}! + n_{i,2}^{(P)}! + \dotsc + n_{i,J_P}^{(P)}!}\right] +
$$
$$
\sum_{i = 1}^{k}\sum_{j=1}^{n_c} \sum_{l=1}^{J_\textbf{S}^{(j)}} \left[ \ln \binom{n_{i,l}^{(j)}+J_{C_j}-1}{J_{C_j}-1} + \ln \frac{n_{i,l}^{(j)}!} {n_{i,1,l}^{(j)}! + n_{i,2,l}^{(j)}! + \dotsc + n_{i,J_{C_j},l}^{(j)}!}  \right]
$$
ahol $\gamma_i$ az i. intervallumba tartozó adatok száma, $J_P$ a változó szülei által felvehető értékek száma (a szülők kardinalitásának szorzata), $n_{i,j}^{(P)}$ a szülő j. esetének számossága az i. intervallumban.
Valamint $n_c$ a változó gyerekeinek száma, $J_S^{j}$ a változó j. gyereke szülei felvehető értékeinek száma, $J_{C_j}$ a változó j. gyereke felvehető értékeinek száma, $n_{i,l}^{(j)}$ a j. gyerek szülei l. esetének számossága az i. intervallumban, $n_{i,r,l}^{(j)}$ pedig a j. gyerek r. esetének és szülei l. esetének számossága az i. intervallumban.

A likelihood tag lényege, hogy a változó szülei, gyerekei és azok szülei lehetséges értékeit figyelembe véve ezekből minél kevesebb típusút minél nagyobb számban tartalmazzon az intervallum. Amennyiben ez teljesül, akkor sikerült az intervallumot úgy meghatározni, hogy a tőle függő változók szerint lett csoportokra bontva.

Elég a szülő, gyerek, gyermeki szülőket vizsgálni, hiszen ez a változó Markov-takarója, amennyiben ezek a változók rögzítettek, akkor a hálóban levő többi változótól független a vizsgált változó (a Markov-takaróján keresztül feltételesen függ a többitől). Így lehet lokálisan diszkretizálni.

\section{Dinamikus programozás}
A \textit{likelihood} tagnál felismerhető, hogy minden tagja $i=1$-től $k$-ig van összeadva, tehát minden intervallumra összegezve van. Ez az összegzés kiemelhető. A $k.$ intervallumba eső összeget, amelynek határai legyenek $\alpha_{k}$ és $\beta_{k}$, jelölje $h(\alpha_{k}, \beta_{k})$. Ekkor a posterior tag $\sum_{i = 1}^{k} h(\alpha_{k}, \beta_{k})$.

Az eredeti minimalizálandó feladat szerint az intervallumok határait keressük. Az első intervallum $X_{1}$-től kell induljon, az utolsónak $X_{n}$-nél kell befejeződni. Minden (kivéve az első) intervallum kezdete az előző vége utáni elem. Ezek a kényszerek a diszkretizációs elvvel szemben támasztott követelmények.

Dinamikus programozási feladattá úgy alakítható át, hogy minden $x_{0}, x_{\psi}$ érték között megállapítjuk a legjobb diszkretizációt, és az arra adott becslést. Sorban haladva az $x$ egyedi értékein ($\delta$) megállapítható, hogy a korábbi pontok közül ($\gamma$) az $x_{0} \dots x_{\gamma}$ becslésének és a $[x_{\gamma + 1}, x_{\delta}]$ becslésének összege mikor maximális, mert ez a diszkretizáció a legvalószínűbb. Fontos, hogy a $x_{0} \dots x_{\gamma}$ nem egyetlen diszkretizációs intervallum, hanem arra már meg van állapítva a diszkretizáció, és el van tárolva az arra adott becslés. A $[x_{\gamma + 1}, x_{\delta}]$ viszont biztosan egy intervallum, hiszen ha lenne egy $\phi \mbox{ ahol } \gamma < \phi < \delta$ határpont, ami jobb értéket ad, azt a $x_{0} \dots x_{\phi}$ vizsgálatakor meg fogjuk találni. A végeredmény az $x_{0}, x_{n}$ között megtalált legjobb diszkretizáció.

Ilyen módszerrel történő keresésnél a $h$ minden két adatpont közötti értéke előre kiszámolható, és eltárolható a $H$ táblában. Ez a dinamikus programozási feladatot meggyorsítja, mert a becslés számolásánál a posterior rész kiolvasható ebből a táblából.

\section{Több változó diszkretizálása}
Ha az adatok közt több folytonos változó van, kezdetben mindegyiket L egyenlő számú adatot tartalmazó intervallumra bontjuk, tehát az \textit{Egyenlő mintaszámú intervallumok elve} szerint diszkretizáljuk,  majd iterálunk a folytonos változókon. Az algoritmus futtatása olyan adatsoron történik, amelyben a kiválasztott változó folytonos, minden más változó diszkrét vagy az aktuálisan legjobb módon diszkretizált. Így a kiválasztott folytonos változóra megkapjuk a jelenleg legmegfelelőbb diszkretizációt.

Az eredeti tanulmány szerint érdemes a gyerekektől indulva iterálni, ám a félév során ezt nem implementáltam, mivel a háló is tanulva volt, így nem csökkentette volna érdemben a feltételezéseket.

Az iterációt addig futtatjuk, ameddig a diszkretizációs elv nem konvergál.

\section{Struktúra tanulás}
A diszkretizációs eljárás a \emph{k2} algoritmust alkalmazza struktúra tanulásához. Ez az algoritmus heurisztikus kereséssel találja meg a legnagyobb valószínűségű Bayes-hálót a diszkrét adatsor alapján \cite{ruiz2005illustration}. Az algoritmus nem robusztus, bemenetként elvárja a változók sorrendjét, ez lesz az elkészült háló topológiai sorrendje.

A k2 algoritmus pontozó függvénye:
$$ f(i, \pi_{i}) =
\prod_{j = 1}^{q_{i}} \frac{(r_{i} - 1)!}{(N_{ij} + r_{i} - 1)!}
\prod_{k = 1}^{r_{i}} \alpha_{ijk}!$$
ahol a $\pi_{i}$ a vizsgált változó szülői halmaza, $\phi_{i}$ (nem szerepel a képletben, de a többi változó magyarázatához szükséges) a vizsgált változó szüleinek összes lehetséges együttesen felvehető értéke, mely a szülők felvehető értékeinek Descartes-szorzata, $q_{i}$ a $\phi_{i}$ elemszáma, $V_{i}$ a vizsgált változó lehetséges felvehető értékei, $r_{i}$ a $V_{i}$ elemszáma, $\alpha_{ijk}$ azon esetek száma, ahol a vizsgált változó a $V_{i}$ k. értékét veszi fel, szülei pedig a $\phi_{i}$ j. értékét veszik fel, $N_{ij}$ = $\sum_{k = 1}^{r_{i}}\alpha_{ijk}$ azon esetek száma, ahol a vizsgált változó szülei a $\phi_{i}$ j. értékét veszik fel.

A megadott sorrendben fut az algoritmus a változókon. Kiértékeli a vizsgált változó aktuális szüleivel az \emph{f} függvény értékét, és ezt tekinti a megtalált legjobbnak. Ezután a sorrendben a vizsgált változó előtt szereplőket lehetséges szülőnek tekinti. Kiértékeli az \emph{f} függvényt minden esetben úgy, hogy a vizsgált változó szülőhalmazához hozzáadja a szülő jelöltet. A kopott értékek közül kiválasztja a legnagyobbat, és ha ez jobb, mint a szülő jelölt nélküli érték, létrehoz egy élt a legjobb szülő jelölt és a vizsgált változó között. Ezután az új szülőhalmazhoz próbál még szülőt adni, amellyel javítható a pontszám. Ha nem talál a pontszámot javító szülőt, vagy a szülők száma eléri az arra vonatkozó korlátot (ezt alacsonyra választva a futási sebesség jelentősen növelhető), a sorrendben következő változóval folytatja.

A tanulmány úgy használja fel a k2 algoritmust, hogy először az egyenlő hosszú diszkretizációs algoritmussal kapott adatsoron futtatja. Az első él hozzáadása után a Bayes-döntés alapú algoritmust futtatja az aktuális hálón, majd az új diszkretizációval kapott adatsoron folytatja a k2 futását. Minden él hozzáadása után újabb diszkretizációt végez, ameddig a k2 algoritmus nem terminál.

A k2 algoritmus kezdeti sorrendjének meghatározására két ajánlatot tettek. Az első, hogy az egyszerű diszkretizációval rendelkező adatokon sok (1000 körüli) k2 futtatást végeztek véletlenszerű sorrendekkel, végül a kapott gráfok közül a legjobbnak a kiindulási sorrendjével futtatták a fő algoritmust. A másik, hogy kevesebb, (20-50) véletlenszerű sorrenddel futtatták a fő algoritmust, és a kapott gráfokból kiválasztották a legjobbat. A különbség, hogy a kiválasztást olyan hálókon végzik, amelyek egyszerű diszkretizáció alapján készültek, vagy már a Bayes-döntés alapján diszkretizált adatsor alapján. A második eset hosszabb futásidőt kíván.

A Diplomamunkában a \ref{chapter:kiertekelesszintetikus}. fejezetben az információelméleti alapon meghatározott sorrenddel futtatás is vizsgálva lesz.
%Implementáció, dokumentáció
%----------------------------------------------------------------------------
\chapter{Bayes-döntés alapján diszkretizáló csomag}\label{chapter:modul}
%----------------------------------------------------------------------------
% végére szekvenciadiagrammok, összehasonlítva az előző fejezet algoritmusával

\iffalse
A Diplomaterv részeként elkészült egy \textit{python} modul\footnote{A python nevezéktanban a függvénykönyvtárakat nevezik moduloknak.}, amely a bemutatott algoritmus használatát könnyíti meg. Az eredeti cikkben bemutatott implementáció \textit{julia} nyelven történt, amely egy gyors és dinamikusan fejlődő nyelv, melyet statisztikai feladatok megoldására gyakran használnak. Az új modul a python nyelven íródott, mert jelenleg ez a nyelv elterjedtebb, és az egyik jelentős előnye, hogy nagy mennyiségű függvénykönyvtárral rendelkezik, és ennek a bővítése segítség lehet a többi felhasználónak.

A modul célja, hogy egy robusztus többváltozós diszkretizációt hajtson végre a kapott adatsoron, valamint a kinyerhető információkat a felhasználó rendelkezésére bocsássa. A \verb|bediscretizer| név a "Bayesian Estimation based DISCRETIZER" rövidítése.

\section{Telepítés}
A modul jelenleg nem telepíthető pip-pel. Az \ref{appendix:codebase} függelékben szereplő linkről letölthető a bediscretizer könyvtár. Ezt a python HOME könyvtárának (például: \verb|C:\Python39|) \verb|Lib| alkönyvtárába bemásolva a modul telepítésre került.

\section{Használat}
Ha a modul telepítve van, a hozzá tartozó python értelmezőben futtatott \verb|import bediscretizer| parancs után felhasználhatóak a biztosított függvények.

A modul központi osztálya a \verb|MultivariateDiscretizer|. Példányosításkor meg kell adni a statisztikai adatsort egy numpy tömbként. Ezen kívül megadható az adatsor neve, amelyet később a kiírásokhoz használ.

Ezután a \verb|fit()| függvényének meghívásával lehet a tanító algoritmust futtatni. Ennek megadható, hány alkalommal futtassa a diszkretizáció és struktúra tanulási algoritmusokat, ez az epochok száma.
A futtatás után kinyerhetőek az adatok a modellből. Egy egyszerű példa látható a \ref{appendix:simpleexample} függelékben.

\section{Elérhető információk}
A \verb|MultivariateDiscretizer| osztályból az adatsorról sok adat kinyerhető. Ezeknek a listája, és a hozzájuk tartozó leírás szerepel itt.

\subsection{Tagváltozók}
\begin{itemize}
    \item[data] A megadott adatsor az előfeldolgozás után. Egy mátrix csak számokkal feltöltve.
    \item[column\_types] Az automatikusan felismert változótípus, amely lehet \verb|DISCRETE| vagy \verb|CONTINUOUS|.
    \item[discretization] A folytonos változókhoz tartozó diszkretizációs intervallumok határai.
    \item[graph] Az adatsorhoz illesztett gráf.
    \item[name] Az adatsor neve, amelyet az adatsorból kinyert adatok kijelzéséhez használ.
\end{itemize}

\subsection{Metódusok}
\begin{itemize}
    \setlength{\itemindent}{5em}
    \item[fit] A tanító algoritmus futását indítja el.
    \item[show] Az aktív \verb|pyplot figure| grafikonjára kirajzolja az illesztett gráfot.
    \item[draw\_structure\_to\_file] A paraméterként megadott fájlba menti az illesztett gráf képét.
    \item[as\_dataframe] Az adatsort \verb|pandas DataFrame| típusúként adja vissza.
\end{itemize}
\fi

Ez a fejezet bemutatja a Diplomaterv részeként elkészült bediscretizer csomag implementációját. Leírja, milyen elvárások, követelmények vannak a függvénykönyvtárral szemben. Ismerteti a mérnöki döntéseket, amelyeket a csomag létrehozásakor kellett meghozni.

A csomag python nyelven lett implementálva. Elsődleges célja, hogy egy robusztus többváltozós diszkretizációt hajtson végre a kapott adatsoron. A \verb|bediscretizer| név a "Bayesian Estimation based DISCRETIZER" rövidítése.

%% spellcheck-off
\section{Követelmények}
A csomaggal szemben támasztott követelmények a FURPS modell \cite{grady1992practical} szerint vannak csoportosítva. Ez egy angol betűszó, amely a csoportok neveinek kezdőbetűiből lett összerakva. Ezek alapján a program különböző szempontokból vizsgálható. A szempontok: funkcionalitás (\textbf{F}unctionality), használhatóság (\textbf{U}sability), megbízhatóság (\textbf{R}eliability), teljesítmény (\textbf{P}erformance) és a támogatottság (\textbf{S}upportability). Egy alkalmazás készítése során ezekre a szempontokra figyelve jobb minőségű végeredmény érhető el, mintha kizárólag a megfelelő működés a cél.
%% spellcheck-on

\subsection{Funkcionális követelmények}
A működés során az elvégzendő feladatokat, a rendszer képességeit, biztonsági kérdéseket válaszolja meg.

\begin{enumerate}
    \descitem{Bemenet} Fogadjon el numpy.ndarray típusú objektumot, amely az adatokat tartalmazza.
    \descitem{Kimenet} Legyen egy betanított Bayes-háló, amely a kapott adatok közötti kapcsolatokat reprezentálja.
    \descitem{Algoritmus} Valósítja meg a \autoref{chapter:bayesdontesalapu}. fejezetben ismertetett Bayes-döntés alapú diszkretizációs eljárást.
    \descitem{Felügyelet nélküli tanulás} Ne várjon címkézett adatokat, kizárólag a bemenet változói közötti összefüggéseket ismerje fel.
    \descitem{Automatikus adattípus} Ismerje fel a diszkrét és folytonos változókat a bemeneti adatsorban, és kezelje ennek megfelelően.
    \descitem{Konfigurálhatóság} A kezdeti konfiguráció megadható legyen, ilyenkor ne végezzen automatikus felismeréseket.
\end{enumerate}

\subsection{Használhatósági követelmények}
A felhasználói élmény javítását célozza, az emberi tényezők figyelembe vételére törekszik.

\begin{enumerate}[resume]
    \descitem{Felhasználók} A csomagot arra kell felkészíteni, hogy fejlesztők és adattudósok használhassák.
    \descitem{Bemeneti típusok} Legyen megadva a \textit{typing} modul segítségével a kezelt típusok. A python gyengén típusos nyelv, ezért ezt a nyelv nem kényszeríti ki, ám a fejlesztői környezetek felismerik és támogatják a típusokat, ezzel segítve a felhasználót.
    \descitem{Logolás} Támogassa a \textit{logging} modult. A logoló beállításait tartsa be és rögzítse a logban a folyamat lépéseit. A kezelt problémákat is jegyezze fel.
    \descitem{Robuszusság - bemenet} Ismerje fel a nem megfelelő bemenetet, és kezelje akár kivételt dobva.
    \descitem{Robuszusság - futás} Algoritmus futása közben fellépő kivételeket kezelje. A helytelen eredmény felismerése nem lehetséges, ezért ez nem is cél.
    \descitem{Ismert metódusok} Hasonlítson kiajánlott metódusok neve és működése a \textit{scikit-learn} csomag algoritmusaihoz. Ezzel segítve a felhasználók könnyebb alkalmazkodását.
    \descitem{Nyílt forráskód} Az implementáció legyen elérhető bárki számára.
\end{enumerate}

\subsection{Megbízhatósági követelmények}
A hibalehetőségek száma, gyakorisága és azok kezelésének lehetőségeit fogalmazza meg.

\begin{enumerate}[resume]
    \descitem{Belső hibák} A belső hibákat kapja el, logolja és csak saját hibákat dobjon. Ezzel elérhető, hogy a helyben kezelhető hibák ne okozzák az alkalmazás futásának a végét.
    \descitem{Determináltság} A futtatott algoritmus legyen determinisztikus
    \descitem{Diszkretizáció hiba} A diszkretizáció során fellépő hibát kezelje úgy, hogy az eredeti diszkretizációt megtartja.
\end{enumerate}

\subsection{Teljesítmény követelmények}
A futási sebességet, hatékonyságot, pontosságot határozza meg

\begin{enumerate}[resume]
    \descitem{Dinamikus programozás} Használja ki, hogy az algoritmusban használt \textit{H} mátrix értéke dinamikus programozással megoldható
    \descitem{Futási idő - referencia} A futási idő legyen a tanulmányban szereplő \textit{Julia} implementációnál rövidebb ugyanarra a bemenetre.
    \descitem{Futási idő - bemenet} A bemeneti adatpontok számának függvényében $O(n^2)$ legyen.
\end{enumerate}

\subsection{Támogatottsági követelmények}
    Tesztelhetőségi, konfigurálhatósági, bővítési lehetőségekre ír le igényeket.

\begin{enumerate}[resume]
    \descitem{Egység teszt} Az algoritmus implementációjához és a használt segédfüggvényekhez legyenek működést ellenőrző tesztek.
    \descitem{Beállítható változó típus} Legyen megadható a bemeneti változók típusa (folytonos és diszkrét)
    \descitem{Beállítható kezdeti diszkretizáció} Lehessen megadni kezdeti diszkretizációt, vagy kiválasztani az ezt létrehozó algoritmust.
\end{enumerate}

\section{Felépítés}
A python nyelven készült függvénykönyvtárakat a nevezéktan két csoportra bontja. Egyik csoport a \textbf{modul}. Ebben az esetben egyetlen fájlban szerepel a kódbázis. Ilyenkor a fájl nevével lehet rá hivatkozni, ezzel emelhető be más programokba. A modulban szereplő minden elem a névtérbe kerül és a használó programban elérhető lesz.

A másik csoport \textbf{csomag}. Ilyenkor egy könyvtárban vannak összegyűjtve a funkciók. A kiajánlott osztályokat és függvényeket a \textit{\_\_main\_\_.py} tartalmazza. Ebben lehet bármilyen python kód, de általában \textit{import} hívások sorozata, amely a csomag fájljaiból, mint modulokból gyűjti össze a kívülről elérhető nyelvi elemeket. Más programokba a könyvtár nevével emelhető be, ilyenkor a kiajánlott funkciók lesznek a névtérben. A python nem támogatja a privát nyelvi elemeket, így ismerve egy csomag felépítését, annak moduljai is beemelhetőek, de ez gyanús kódot (code smell) eredményez.

A csomag a \textit{MultivariateDiscretizer.py} modul köré épül. Megvalósítható lett volna egyetlen modulban is, de a könnyebb bővíthetőség és olvashatóság érdekében több modulra lett bontva. Így később akár más nagyobb diszkretizációs eljárások is könnyen hozzáadhatók.

A fő modul, amely az algoritmust végrehajtó osztályt tartalmazza a \textbf{MultivariateDiscretizer}. Ebben található az azonos nevű osztály, mely példányosításkor várja az adatsort, melyből automatikusan képes beállítani a konfigurációját, valamint egyéb konfigurációs paraméterek is megadhatók. Ezen az elven működnek a \textit{scikit-learn} csomag implementációi is. A tanulás megkezdéséhez a \textit{fit} metódust kell meghívni, ez szintén azonos a hasonló csomagok használatával. Azért szükséges külön bontani az inicializálást és a tanulást, mert így mérhető a sebességük külön-külön is a felhasználó számára.

A Bayes-döntés alapú diszkretizációs eljárás implementációja a \textbf{discretization} modulban szerepel. Azért szerepel külön, mert így könnyen tesztelhető fejlesztés közben, hiszen futtatható a fájl önmagában is. Emellett több különböző implementációja is elkészült az algoritmusnak, és itt ezek együtt szerepelhetnek.

Egyéb segédfüggvényeket tartalmaz a \textbf{util} modul. Ezek nem szorosan kapcsolódnak az eljáráshoz, egyéb esetekben is hasznosak lehetnek, így ki lettek szervezve egy külön modulba. Ennek az előnye, hogy könnyen írható ezekre teszteset, hiszen a bemeneti paramétereik alapján determinisztikusak. A \textit{MultivariateDiscretizer} metódusai az osztályváltozókat is használják, így ezek teszteléséhez példányosítani kell az objektumot.

Az implementált struktúra tanulási algoritmusok és ehhez segédfüggvények a \textbf{structure} modulban kaptak helyet. Emellett a \emph{pomegranate} implementációkhoz is biztosít egy csomagoló függvényt, hogy minden elérthető struktúra tanulási mód egységesen elérhető és használható legyen.

A \textbf{test} könyvtárat nem tartalmazza a csomag, de a forráskód mellett elérhető. Ezek a függvények tesztelik az implementáció működését.

\begin{figure}[htp]
    \centering
    \includegraphics[width=12cm]{figures/app/structure.png}
    \caption{A rendszer felépítése}
    \label{fig:struktura-diagram}
\end{figure}

% TODO: tesztelési környezet hozzáadása

\section{Működés}
A csomag elsődlegesen a \emph{MultivariateDiscretizer} osztályon keresztül használható. Fő funkciói a példányosítás, a tanítás és a kiértékelés. Az folyamatok során elvégzett lépések leírását tartalmazza ez a rész.

\subsection{Példányosítás}
A használathoz először a \emph{MultivariateDiscretizer} osztályt kell példányosítani. Paraméterei között kötelező a tanító adatsort megadni ndarray típusú mátrixként. Erre azért van szükség, hogy az alapértelmezett belső paramétereket ki lehessen számolni. Ezen kívül megadható az adatsor neve, amely a generált eredmények elnevezéséhez használható, a struktúra tanuláshoz használandó algoritmus mely értéke \emph{greedy, chow-liu, exact, k2 vagy multi\_k2} lehet, alapértelmezett a multi\_k2, amely a Bayes-döntés alapú diszkretizáció eljárást futtatja a később megadott számú alkalomszor és a legjobbat kiválasztja. Felülírható az alapértelmezetten automatikusan eldöntött változó típusok, valamint a kiindulási Bayes-háló struktúrája. Kiválasztható a kezdeti diszkretizációhoz alkalmazott algoritmus, értéke \emph{equal\_width vagy equal\_sample} lehet, melyek rendre az egyenlő hosszú és az egyenlő mintaszámú eljárások.

A példányosítás során a kapott adatsorban megvizsgálja, hogy van-e olyan változó, mely értékei szövegesek. Ezek a változók diszkrétek ugyan, de a későbbi számolásokat megkönnyítik a számértékek, ezért a szöveges változó minden lehetséges értékéhez egy számot rendel, az adatsorban ezzel helyettesíti. A dekódoláshoz az objektum \emph{column\_unique\_values} attribútumában megmaradnak a hozzárendelések, ez bármikor elérhető.

Amennyiben a változók típusa nincsen megadva, automatikusan felismeri a típusokat. Ehhez a változókon egyesével ellenőrzi, hogy a változó összes értéke egész szám-e. Amennyiben igen, az osztályban definiált \emph{ColumnType} enum \emph{DISCRETE} értékét állítja be rá, ha nem, akkor pedig a \emph{CONTINUOUS} értéket. Az így kapott lista a \emph{column\_types} attribútumban érhető el. Hasonló formában kell megadni a példányosításkor, ha nem az automatikus felismerés a cél.

Ezután a beállított kezdeti diszkretizációt végzi el. Először a diszkrét változókon végig haladva mindegyik kardinalitását megállapítja. Ezt a numpy \emph{unique} \footnote{https://numpy.org/doc/stable/reference/generated/numpy.unique.html} függvény segítségével teszi, mely visszaadja az egyedi értékeket a változóban. Ezek száma pontosan a kardinalitás. A legnagyobb számosságot eltárolja a \emph{number\_of\_classes} attribútumban.

A folytonos változókon végig haladva a beállított kezdeti diszkretizáció alapján meghatározza a diszkretizációs elvet, majd a \emph{discretization} attribútum megfelelő elemeként eltárolja. Ez az attribútum egy lista, mely minden változóhoz tartalmazza a rajta használandó diszkretizációs elvet, ami a vágási értékék listája. A diszkrét változóknál a diszkretizációs elv \emph{None}.

Az egyenlő mintaszámú diszkretizáció során a kiszámolja, mely sorszámú elemek lesznek a határpontok. Ezek az \emph{adatok száma} / \emph{number\_of\_classes} hányados egészre kerekített értéke helyeken állnak. A változó rendezett értékeinek listájából kiemeli ezeket az elemeket, listába rendezi, ez lesz a diszkretizációs elv az adott változóra.

Az egyenlő hosszú intervallumok esetében a változó értékeinek minimumát és maximumát keresi ki. A \emph{min + (max - min)} / \emph{number\_of\_classes} értékek adják ebben az esetben a diszkretizációs elvet.

A kezdeti diszkretizáció meghatározása után, amennyiben nincsen kezdeti háló struktúra megadva, a beállított struktúra tanuló algoritmussal, valamint a diszkretizációs elv által meghatározott diszkrét adatsorral fut a struktúra tanulás. Ezt azért végzi el, hogy legyen egy kiindulási struktúra. A kezdeti struktúra a \emph{graph} attribútumban van eltárolva, \emph{networkx.DiGraph} \footnote{https://networkx.org/documentation/stable/reference/classes/digraph.html} típusú objektumként.

\subsection{Tanítás}
A példányosítás után a kitöltött attribútumokat meg lehet vizsgálni, felül lehet írni, ezért a tanítás külön lépésben szerepel. Emellett ha nem szükséges a diszkretizáció - struktúra tanulás lépések ismételgetése, (például egyváltozós diszkretizációt kell csak alkalmazni,) a példányosítás után rendelkezésre is áll a betanított Bayes-háló, nem szükséges semmilyen tanító lépést meghívni.

A tanítás a \emph{fit} függvény meghívásával indítható. Paraméterként a maximális futtatások számát várja. Ennek jelentése algoritmusonként eltér, de arra szolgál, hogy a meghatározott számú iteráció után mindenképp leálljon, ne kerüljön végtelen ciklusba. A meghívás után a beállított struktúra tanuló algoritmus alapján folytatódik a futás.

A \emph{k2} beállítás egy teljes k2-es tanítást futtat a \ref{chapter:bayesdontesalapu}. fejezetben leírt algoritmus alapján. Ez elindítható a \emph{fit\_k2} metódussal is, mely paraméterként a változók sorrendjét várja. Ennek megadására a \emph{fit} függvény esetében nincsen lehetőség. Ebben az esetben véletlenszerű sorrenden történik a futtatás. A struktúra tanulásához a \emph{structure} modul \emph{learn\_structure} függvényét \emph{p\_step} paramétert megadva hívja fel, ezért az a k2 algoritmust a paraméter értékét felhasználva folytatja, és egy új él megtalálása után azonnal visszatér az új p\_step-pel. Így érhető el, hogy azonos interfészen megvalósulhasson a struktúrának részleges és teljes tanulása is.

A struktúra tanulás után meghívja a belső \emph{\_discretize\_all} metódust, mely eltárolja a jelenlegi diszkretizációt. Ezután a jelenlegi gráf topológiai sorrendjében visszafele haladva minden folytonos változóra meghívja a \emph{\_discretize\_one} metódust, mely frissíti annak a változónak a diszkretizációs elvét. Majd miután az összes frissült, ellenőrzi, hogy változtak-e az eltárolt eredetihez képest. Amig történik változás, de legfeljebb 10 alaklommal, ezt a három lépést ismétli.

Az adott változó diszkretizációs elvének frissítéséhez a \emph{util} modul \emph{largest\_markov\_cardinality}, majd az ebből kapott kardinalitással a \emph{discretization} modul \emph{discretize\_one} függvényének felhívása történik. Erről részletesebben a modulok leírásánál lesz szó.

A k2 tanítás a struktúra tanulás egy lépését, majd teljes diszkretizációt futtat, ameddig a struktúrába kerül új él. Ezután a kialakult modellt eltárolja a \emph{final\_model} attribútumban, és a tanítás visszatér.

A \emph{multi\_k2} algoritmussal ellenőrzi, hogy a változók permutációinak száma meghaladja-e a megadott maximális futtatások számát. Amennyiben nem, minden permutációban futtatja az előzőekben ismertetett k2 tanítást, másként a lehetséges sorrendekből kiválasztja a megengedett számút, és ezekkel futtatja a k2 tanítást.

Minden sorrendű futás után értékeli a felépített hálót a \emph{util.preference\_bias\_full} függvénnyel, és az ez alapján legjobbnak ítéltet a végén újra betanítja. Mivel a k2 tanítás meghatározott sorrend mellett determinisztikus, ugyanaz lesz az eredmény, mint ami a legjobb gráfot létrehozta. Az újra tanítás azért szükséges, mert így elegendő a változó sorrendet tárolni, nem kell minden más információt is.

A \emph{chow-liu, greedy, exact} paraméterek esetén két metódus is implementálva van, de ezek nem adnak vissza értelmezhető eredményt. Az egyik a \emph{\_fit\_basic\_structure\_learner}, amely a Friedman és Goldszmidt \cite{friedman1996discretizing} által meghatározott algoritmus szerint tanítja a hálót. Ez alapján egymás után futtatja a Bayes-döntés alapú diszkretizációt és a megadott struktúra tanulást addig, amíg a diszkretizáció nem konvergál. Ez a struktúra konvergenciával egyszerre történik meg, mert ez a diszkretizációs algoritmus determinisztikus. A probléma az volt, hogy a struktúra tanuló algoritmusok olyan hálót készítettek, amely alapján a diszkretizáció gyakran üres halmazt talált csak megfelelőnek, tehát a változók minden értékét ugyanabba az osztályba sorolta. A tanulmányban megadott Julia-ban készített referencia implementáció is ugyanígy viselkedett ezeken a struktúrákon, úgyhogy ez nem implementációs hiba.

A másik metódus a \emph{\_fit\_basic\_structure\_after\_node\_added}, mely a k2 tanulást utánozza a beépített struktúra készítőkkel. Nyilvántart egy külön gráfot, mely az elején üres. Minden struktúra tanítási lépésben az algoritmusnak megadja kényszerként, hogy ezt a gráfot tartalmaznia kell. Az algoritmus által készített háló minden élét, amellyel még DAG marad, hozzá próbálja adni a nyilvántartott gráfhoz. Ezeket pontozza a multi\_k2-ben megismert módon, és a legjobbat megtartja, így mindig eggyel növelve a gráf éleinek számát. Ezután végzi el a diszkretizációt, egészen addig, míg a struktúra tanuló algoritmus nem növeli az élek számát. Ezt futtatva ugyanaz lett az eredmény, mint a \_fit\_basic\_structure\_learner esetén, mert ezek a struktúra tanuló algoritmusok nem voltak érzékenyek a diszkretizációra, ugyanazt a gráfot adták vissza minden esetben. Ez az Iris adatsoron történt, lehet, hogy más adatsorral jobban működnének.

A \emph{best\_edge} nevet algoritmusnak megadva szintén a k2 tanulás egy variánsa fut. Ebben az esetben nem a meghatározott sorrend alapján megy végig a változókon, hanem az összes lehetséges élt próbálja beilleszteni a nyilvántartott gráfba, majd a multi\_k2-ben megismert módon pontozottakból a legmagasabb pontszámot elérőt illeszti be végül. Véletlenszerű sorrenden futtatott k2 algoritmusnál nem ért el jobb eredményt az Iris adathalmazon. Emellett fennáll a veszélye, hogy nagy adathalmaznál a lehetséges élek nagy száma miatt jelentősen lelassul.

\subsection{Kiértékelés}
A kiértékelés során a modell tesztelése történik. A tanítás után egy tesztelő adatsorral (amely lehetőség szerint olyan adatpontokat tartalmaz, melyek a tanító adatsornak nem voltak részei) predikciót készít a modell. A predikált értékek pedig összehasonlíthatók a valódi értékekkel. Az összehasonlításra különböző módszerke vannak. A Bayes-hálók esetében, mivel diszkrét adatokkal dolgoznak, a klasszifikációs kiértékelési módszerek alkalmazhatóak, tehát a predikció az adatpontokról megállapítja, hogy a célváltozó melyik osztályába tartoznak, a kiértékelés pedig konfúziós mátrix alapján történik.

A predikcióhoz a \emph{MultivariateDiscretizer predict} metódusa alkalmazható. Ennek paraméterként a teszt adatsor kell megadni, valamint a célváltozó azonosítóját. Ezután a tárolt diszkretizációs elv szerint a teszt adatokat is diszkretizálja, ismeretlenre állítja a célváltozó értékeket, majd ezt az adatsort a tárolt betanított modell \emph{predict} \footnote{\url{https://pomegranate.readthedocs.io/en/latest/BayesianNetwork.html\#pomegranate.BayesianNetwork.BayesianNetwork.predict}} metódusának átadja. Az eredményből visszaadja a célváltozóra kapott predikált értékeket.

A konfúziós mátrix elkészítésére az \emph{evaluate} metódus szolgál. Ennek bemenetként a valós és a predikált értékekre van szüksége, emellett megadható a korábbi kiértékelések eredménye is. Ha mag van adva, akkor a korábbiak végéhez fűzi a számolt értékeket, ha nincsen akkor egy új eredmény objektumot hoz létre. Többosztályos változó esetén minden változóra kiszámolja a scikit-learn \emph{confusion\_matrix} \footnote{https://scikit-learn.org/stable/modules/generated/sklearn.metrics.confusion\_matrix.html} függvénye által kiadott konfúziós mátrixból a változó bináris konfúziós mátrixát, ennek értékeit teszi az eredmény objektumba. A kiértékelés metrikái bináris konfúziós mátrixon működnek, ezért van szükség az átalakításra.
%https://stackoverflow.com/questions/31324218/scikit-learn-how-to-obtain-true-positive-true-negative-false-positive-and-fal
%TODO: konfúziós mátrix az elejére

A metrikák alkalmazásához az \emph{evaluation\_summary} metódus használható. Ez az előzőekben elkészített eredmény objektumot várja, annak értékeit összegzi, így megkapja hogy az összes teszt során összesen hány darab valódi pozitív (TP), hamis pozitív (FP), valódi negatív (TN) és hamis negatív (FN) eredmény lett. A metódus kulcs-érték párokként visszaadja a k2 tanításával foglalkozó tanulmányban \cite{aghdam2019some} szereplő metrikák eredményeit. Ezek a valódi pozitív, hamis pozitív, valódi pozitív arány, hamis felfedezési arány, pozitív prediktív érték, pontosság, Matthews korrelációs együttható és az F-pontszám. Ezek mindegyike kiszámolható a kapott négy adatból, és összehasonlíthatóvá teszi a Bayes-hálókat.

Valódi pozitív arány $TPR = \frac{TP}{TP + FN}$, hamis felfedezési arány $= \frac{FP}{FP + TP}$,
pozitív prediktív érték $PPV = \frac{TP}{TP+FP}$, pontosság $= \frac{TP + TN}{TP + FP + TN + FN}$,
Matthews korrelációs együttható $= \frac{TP \cdot TN-FP \cdot FN}{(TP + FP)(TP + FN)(TN + FP)(TN + FN)}$,
F-pontszám $= 2 \cdot \frac{PPV \cdot TPR}{PPV + TPR}$.

%TODO: szekvencia diagram

\section{Modulok}
Ez a rész tartalmazza MultivariateDiscretizer melletti modulok implementációinak leírását. Ezekben nincsen osztály definiálva, csak kiajánlott függvények, amelyek a csomag elsődleges funkcióján kívül is használhatók, rendezett csoportokban.

\subsection{Util}
A \emph{discretize} függvény az adatsorból és a diszkretizációs elvből elkészíti a diszkretizált adatsort. Ehhez a DataFrame-ként kapott adatsorban egy segédoszlopot hoz létre. A diszkretizációs elvvel rendelkező oszlopokon megy végig, az elv minden határánál nagyobb értékű elemeket a határpont sorszámának megfelelő osztályba helyez a segédoszlopban, majd ezt visszamásolja az eredeti helyére.

A \emph{bn\_to\_graph} a pomegranate.BayesianNetwork típusú modellből készít DiGraph típusú gráfot. A modell struktúrája minden változóhoz tartalmazza a szülőváltozóit, ezen iterálva felveszi az éleket az új gráfba.

A \emph{graph\_to\_bn\_structure} az előző inverze, az adott gráf struktúrából készíti el azt a modellt, melyet a pomegranate struktúra tanuló algoritmusai értelmezni tudnak.

A \emph{parents\_to\_graph} a k2 algoritmus által készített listából, mely minden változóhoz a szüleit tartalmazza, készíti el a gráfot bn\_to\_graph-hoz hasonlóan.

A \emph{show} függvény a Bayes-hálót rajzolja ki. Átalakítja gráffá, amit a networkx csomag gráfrajzoló függvénye jelenít meg.

A \emph{concat\_array} két numpy tömböt egyesít oly módon, hogy ha valamelyik dimenzióban egyforma darabszámú elemet tartalmaznak, annak mentén illeszti az adatsort. Az adatsorokban gyakran külön változóban vannak a tanító és a célváltozók, ezek illesztéséhez hasznos.

\begin{figure}[htp]
    \centering
    \includegraphics[width=8cm]{figures/app/markov-blanket.png}
    \caption{A D változó Markov-takarója}
    \label{fig:markov-blanket}
\end{figure}

A \emph{markov\_blanket} egy gráf és egy változó azonosítója alapján kikeresi a változó Markov-takarójába tartozó változókat. Ebbe a vizsgált változó szülei, gyerekei és gyerekeinek a más szülei tartoznak bele. Ezek rögzített értéke mellett a változó független az összes többi változótól. Egy példa a \ref{fig:markov-blanket}. ábrán látható.

A \emph{largest\_markov\_cardinality} végigmegy a kapott változó Markov-takaróján, és kiválasztja a legnagyobb kardinalitású számosságát.

A \emph{preference\_bias} a k2 algoritmust pontozó algoritmus. A pontszámot ez az \emph{f} függvény adja, ami a \ref{chapter:bayesdontesalapu}. fejezetben ki van fejtve:
$$ f(i, \pi_{i}) =
\prod_{j = 1}^{q_{i}} \frac{(r_{i} - 1)!}{(N_{ij} + r_{i} - 1)!}
\prod_{k = 1}^{r_{i}} \alpha_{ijk}!$$
Amennyiben a változó szülőinek száma 0, akkor létrehoz egy olyan szülőt, melynek minden értéke egyforma. Ez a képlet alkotójának is az elképzelése \cite{cooper1992bayesian}. Az implementáció a numerikus stabilitás érdekében a függvény logaritmáltját számolja, így a szorzatból összeg lesz. Ez az eredményeken nem változtat, mivel a természetes alapú logaritmus szigorúan monoton növekvő, és a kapott pontszámok csak egymással vannak összehasonlítva. A faktoriális logaritmusa emellett átírható $\Gamma$ függvényes alakba.
$$ ln(n!) = ln(\Gamma(n + 1))$$
Ennek segítségével a scikit-learn \emph{gammaln} \footnote{https://docs.scipy.org/doc/scipy/reference/generated/scipy.special.gammaln.html} függvénye is használható a pontszám kiszámítására. Ez pedig a numpy mátrixokon sokkal nagyobb sebességgel képes a műveletet elvégezni, mint egy elemenkénti megvalósítás \cite{smith2015cython}, mert a háttérben futó \emph{C} nyelven írt kódban történik az iteráció. Ezért az implementációban nagyon fontos szerepet kapott, hogy minden műveletet a numpy és scikit-learn függvények végezzenek, a függvény kódjában a lehető legkevesebb \emph{for} ciklus szerepeljen.

A \emph{preference\_bias\_full} a teljes gráfnak határozza meg az értékét a k2 pontozó algoritmusa alapján. Minden változóra meghívja az előző pontozást, az eredményeket pedig összeadja. Azért elegendő ez a művelet, mert a pontszám logaritmáltját adja vissza. Ez az összeg a gráf értéke.

Az \emph{entropy} egy diszkrét adatsor változója Shannon entrópiájának kiszámolására szolgál. Ennek értéke:
$$ H(\mathcal{X}) = -\sum_{i = 1}^{n} p(x_{i}) \cdot log(p(x_{i}))$$
tehát a diszkrét változó értékének előfordulási valószínűségei és logaritmáltjaik szorzatának összege. Az implementáció természetes alapú logartimust használ, az eredmények összehasonlításra szolgálnak, így bármilyen egynél nagyobb logaritmus alap megfelelő. A sebesség itt is lényeges, így numpy függvényekkel vannak a számítások megoldva. Együttes entrópia számolására is képes, amennyiben tömb helyett mátrixot kap bemenetként.

A \emph{relative\_entropy} egy diszkrét adatsor több változója relatív Shannon entrópiájának kiszámolására szolgál. Ez megkapható:
$$ H(\mathcal{Y} | \mathcal{X}) = -\sum_{i = 1}^{n}\sum_{j = 1}^{m} p(x_{i}, y_{y}) \cdot log(\frac{p(x_{i}, y_{y})}{p(x_{i})})
= H(\mathcal{Y}, \mathcal{X}) - H(\mathcal{X})$$
Az implementációban az előző függvény meghívása történik először az együttes entrópia kiszámolására, majd a feltétel entrópiájának számolására. Ezért a feltételben itt is lehetséges több változó szerepeltetése.

A \emph{mutual\_information} a kölcsönös információt számolja ki az egyik változó entrópiája és a másikkal vett feltételes entrópia különbségeként. AZ ezt használó algoritmus vagy lassan fut vagy nem determinisztikus, ezért ez nem volt használva.

\subsection{Structure}
A \emph{learn\_structure} egységesíti a pomegranate beépített struktúra tanuló algoritmusai, valamint a k2 algoritmus futtatását. A k2 esetén a struktúra tanítás utána a pomegranate \emph{from\_structure} \footnote{\url{https://pomegranate.readthedocs.io/en/latest/BayesianNetwork.html\#pomegranate.BayesianNetwork.BayesianNetwork.from\_structure}} algoritmussal készíti el a Bayes-hálót. Így minden esetben az elkészült és betanított Bayes-hálóval tér vissza, de egyedi struktúra tanulást is lehetővé tesz.

A \emph{k2\_order\_mutual\_information} és a \emph{k2\_order\_entropies} egyaránt Aghdam et al. \cite{aghdam2019some} cikkében szereplő algoritmusok implementációja. Ezek információelméleti alapon keresik ki a legmegfelelőbb, k2 tanuláshoz szükséges topológiai sorrendet. Az első esetében a kezdő változó az eredeti leírásban nincsen meghatározva, az implementációban a legkisebb entrópiájú változó. Ezután a legutóbb hozzáadott változóval vett legnagyobb együttes információval rendelkező változót adja a sorrend végéhez. Mivel a kezdő változó nem meghatározott, az algoritmus használatára nem került sor. A második függvény relatív entrópiák szerinti sorrend eredménye a \ref{chapter:kiertekelesszintetikus}. és \ref{chapter:kiertekelesvalos}. fejezetekben megjelenik. Ez az eredeti algoritmus szerint is a legkisebb entrópiájú változóval kezd, majd mindig azt adja a sorrend végéhez, amelyikhez a feltéve a korábbi változókat a feltételes entrópia a legalacsonyabb. \emph{A, B, C} változók esetében az algoritmusok így futnak:

A kölcsönös információ alapúnál legyen B-ből indítva. Ekkor az $MI(B, A)$ és az $MI(B, C)$ közül a nagyobbat választjuk, most legyen ez az $MI(B, A)$. Az $MI$ a kölcsönös információ meghatározására szolgáló függvény. Ekkor az $A$-t adjuk a sorhoz, és az $MI(A, C)$ közül kell választani. Mivel ez egyedül van, így a $C$ is hozzáadódik, a végső sorrend \emph{B, A, C}.

A feltételes entrópia alapúnál keressük a $H(A)$, $H(B)$ és $H(C)$ közül a legkisebbet. Ez legyen a $H(C)$, így a $C$ a sorrend első eleme. Ezután a $H(A|C)$ és $H(B|C)$ közül kell a legkisebb. Ez legyen a $H(A|C)$, így az $A$ a következő elem. Végül a $H(B|A,C)$ közül a legkisebb az egyetlen, így $B$ az utolsó elem. A végső sorrend \emph{C, A, B}.

A \emph{learn\_k2\_structure} a k2 algoritmus implementációja. Annyi bővítés szerepel benne a \ref{chapter:bayesdontesalapu}. fejezetben ismertetetthez képest, hogy megadható a p\_step paraméter, amely az aktuális gráfállapotot tartalmazza. Ilyenkor ebből az állapotból folytatódik az algoritmus futása és új él hozzáadása után azonnal visszatér.

\subsection{Discretization}
Ez a modul tartalmazza a Bayes-döntés alapú diszkretizációs algoritmust. A MultivariateDiscretizer által használt függvény a \emph{discretize\_one}. Ez bemenetként megkapja a teljes diszkrét adatsort, a modell hálóját, a vizsgált változó folytonos adatsorát, valamint a célt az intervallumok számára (\emph{L}). A dinamikus programozással előre kiszámolható \emph{H} táblát (mely az ismertetett $h$ értékeket tartalmazza minden adatpont között) elkészítő függvényt felhívja. Erre 3 különböző implementáció is készült, amely a későbbiekben lesz bemutatva. A prior tag kiszámolása numpy tömbműveletekkel történik, mert is lényeges a futási sebesség. Ezután a tanulmányban leírt módon történik az összegzés és a diszkretizációs határok keresése a prior és a posterior tag alapján.

A \emph{H} tábla kiszámítására az első függvény a \emph{precalculate\_probability\_table\_as\_definition}. Ez a \emph{H} táblát úgy állítja elő, hogy végigmegy a folytonos értékekből képzett párokon, és képlet alapján mindegyikre kiszámolja az eredményt, majd ezeket eltárolja. Ez nagyon lassú folyamat, mert minden párnál a köztük lévő adatokat ki kell olvasni az adatsorból és processzor intenzív számításokat végezni (logaritmus, faktoriális).

Ennél jobban működik a \emph{precalculate\_probability\_table\_split\_up}. Ez a szülők, valamint a gyermekek és gyermeki szülők felvett értékeinek darabszámát minden \emph{x} érték esetében egy külön táblába gyűjti. Minden párnál az eggyel kisebb intervallumra számolt darabszámokhoz hozzáadja az újhoz tartozó darabszámot, és ezt is eltárolja. Így minden párhoz megkapja a köztük lévő intervallumon felvett értékek darabszámait viszonylag kevés számítással. A \emph{h} kiértékelésekor elég az átmeneti táblából kiolvasni a felvett értékek darabszámát, nem kell minden egyes párnál újra kiszámolni.

Az implementáltak közül a \emph{precalculate\_probability\_table\_split\_up\_numpy} a legjobban optimalizált algoritmus. Ez a $h$ kiértékelését nem \emph{for} ciklusokkal végzi, hanem a numpy és scikit-learn függvényeit használja. Az implementáció az előző függvény alapján készült, minden \emph{for} ciklusnál a benne szereplő függvényeknek meg kellett találni a numpy implementációját, az átadott tenzorokat megfelelő formára hozni, és minden lépést külön elvégezni.

Az eredmények az eredeti tanulmányhoz mellékelt \emph{data\_auto\_mpg.csv} alapján voltak értékelve. Ennek az 5. oszlopán történt a vizsgálat, melynek a példa teszt szerint a 3. oszlop a szülője, az 1. és a 7. pedig a gyereke. A definíció szerinti $H$ számítással a diszkretizáció \textbf{3 perc 9.717 másodpercig}  tartott. Ugyanez a külön táblában tárolónál \textbf{2 perc 32.736 másodpercet vett igénybe}. Végül a numpy optimalizációval készült $H$ táblával \textbf{3.664 másodperc} kellett a diszkretizációhoz. A julia-ban implementált referencia algoritmusnak ugyanehhez \textbf{14.543 másodperc}re volt szüksége, mert ott nem áll rendelkezésre hasonló optimalizálási lehetőség. Ez egyetlen változó egyetlen diszkretizációja, a teljes algoritmus futása ennél hosszabb, de az arányok ott is megmaradnak. Ennek oka, hogy a $H$ tábla előállítása a folyamat legidőigényesebb része. Azért az 5. változó volt vizsgálva, mert ez rendelkezik szülővel és gyerekkel is, amely a kódlefedettséget biztosítja.

\section{Tesztelés}
A tesztek a \emph{test} könyvtárban találhatók, a \emph{unittest} modult felhasználva készültek. A \emph{test.py} tartalmazza a MultivariateDiscretizer-t és a diszkretizációt ellenőrző teszteket. Ezek megállapítják, hogy megfelelően tölti-e be az adatokat, valamint a példa adathalmazon elvégzett diszkretizáció helyes-e. A \emph{unit\_test.py} ellenőrzi, hogy az util diszkretizációja megfelelően működik, valamint a Markov-takarót is meg tudja határozni.

A tesztek futtatásához a gyökérkönyvtárban kell kiadni a követkető parancsot:

\begin{lstlisting}
    python -m unittest test.unit_test
    python -m unittest test.test
\end{lstlisting}

Ez meghívja az unittest modult a test könyvtárban található fájlokra, és futtatja a teszteket.
%Tesztelés szintetikus adatokon - képek a bemeneti adatokhoz, összehasonlítás más algoritmusokkal, sebesség mérés
%%Tesztelés valós adatokon
%Értékelés
%Összefoglalás
%----------------------------------------------------------------------------
\chapter{Összefoglalás}
%----------------------------------------------------------------------------
A Diplomamunka célja a jelenlegi diszkretizációs eljárások bemutatása, egy új eljárás implementálása és ennek kiértékelése szintetikus és valós adatokon volt.

Először ismertette a szükséges alapismereteket, kezdve a statisztikai elnevezésektől a bayesiánus valószínűségelmélet bemutatásán át a javasolt algoritmusig. Az új eljárást bemutatta, valamint a lépéseinek értelmezte az okait is.

Ezután az elkészült implementáció következett. Ebben ismertette az implementálás során szem előtt tartott célokat. Bemutatta az implementáció felépítését és működését. Majd a belső modulokban elérhető függvények ismertetése történt meg. Végül kiértékelte az implementáció eredményét, annak az elsődleges célja sikeres volt, mivel gyorsabban fut, mint az eredeti tanulmányban megjelent referencia. Emellett robusztusabb is és másfajta mérések is végezhetők a csomaggal.

Végül a különböző adatsorokkal való kiértékelés következett. Elsőként a szintetikus adatokon látható volt, hogy az egyszerű adatsoroknál egy egyváltozós diszkretizációs algoritmus is tud olyan eredményt elérni, mint a javasolt algoritmus. Másodikként egy összetettebb adatsoron megismerhetővé vált az előnye. Itt sikerült olyan eljárást találni a bemeneti topológiai sorrend meghatározására, mellyel hasonló eredmény elérhető, mint a referencia, de nagyobb sebességgel. Harmadikként egy szív- és érrendszeri kockázatot vizsgáló modellen kiderült, hogy a laboratóriumi diagnosztikában használt referencia intervallumok kevéssé határozhatók meg a módszer segítségével. Negyedszer az is bebizonyosodott, hogy a laboratóriumi paraméterekre is tud olyan diszkretizációt meghatározni, mellyel a Bayes-háló alapú predikció működőképes.

\section{Továbbhaladási lehetőségek}
A k2 algoritmusnak szüksége van egy topológiai sorrendre a futáshoz. Ennek meghatározása információelméleti módszerekkel bemutatásra került és működőképesnek bizonyult. Ám az információ entrópia meghatározásához a kezdeti diszkretizációs elv alapján készült diszkrét adatsor volt szolgáltatva, mert az ehhez szükséges algoritmus leírása diszkrét adatokat várt el. Ez érthető, hiszen a Bayes-háló struktúrájának meghatározásához általában ilyen adat áll rendelkezésre. Ám jelenleg pont a diszkretizálás a cél, ezért rendelkezésünkre áll magasabb információ tartalmú folytonos adatsor is. Az információelméletben folytonos adatsoroknak is meghatározható az entrópiája, ezért lehetne próbálkozni a folytonos adatokon meghatározott entrópiával is, valamint az így meghatározott feltételes entrópiával és együttes információval. A nehézség, hogy ekkor vegyesen lennének benne folytonos és diszkrét változók, de lehetséges, hogy így jobb topológiai sorrend lenne elérhető.

Másik irány az implementációnál bemutatott, jelenleg nem használt algoritmusok ötletének fejlesztése. A referencia algoritmusban a k2-es élhozzáadás után diszkretizáció következik. Más struktúra tanuló algoritmusoknál is tesztelhető, hogy új él után diszkretizációs lépést futtassunk. A Diplomamunka során implementált ilyen típusú próbálkozás rossz működésének oka lehet, hogy mindig csak a teljesen betanított hálóból tudott élt átvenni. Saját implementációnál elérhető lenne, hogy egy él megtalálása után azonnal visszatérjen, hogy lehetőséget adjon a diszkretizációra.

%\listoffigures\addcontentsline{toc}{chapter}{Ábrák jegyzéke}
%\listoftables\addcontentsline{toc}{chapter}{Táblázatok jegyzéke}

\printbibliography[title={Irodalomjegyzék}]
%\bibliography{mybib}
\addcontentsline{toc}{chapter}{Irodalomjegyzék}
%\bibliographystyle{plain}

%----------------------------------------------------------------------------
\appendix \label{chapter:függelék}
%----------------------------------------------------------------------------
\chapter*{Függelék}\addcontentsline{toc}{chapter}{Függelék}
\setcounter{chapter}{6}  % a fő fejezet-számláló az angol ABC 6. betűje (F) lesz
\setcounter{equation}{0} % a fő fejezet-számláló az angol ABC 6. betűje (F) lesz
\numberwithin{equation}{section}
\numberwithin{figure}{section}
\numberwithin{lstlisting}{section}
%\numberwithin{tabular}{section}


%New colors defined below
\definecolor{codegreen}{rgb}{0,0.6,0}
\definecolor{codegray}{rgb}{0.5,0.5,0.5}
\definecolor{codepurple}{rgb}{0.58,0,0.82}
\definecolor{backcolour}{rgb}{0.95,0.95,0.92}

%Code listing style named "mystyle"
\lstdefinestyle{mystyle}{
  backgroundcolor=\color{backcolour},   commentstyle=\color{codegreen},
  keywordstyle=\color{magenta},
  numberstyle=\tiny\color{codegray},
  stringstyle=\color{codepurple},
  basicstyle=\ttfamily\footnotesize,
  breakatwhitespace=false,
  breaklines=true,
  captionpos=b,
  keepspaces=true,
  numbers=left,
  numbersep=5pt,
  showspaces=false,
  showstringspaces=false,
  showtabs=false,
  tabsize=2
}

\section{Kódbázis} \label{appendix:codebase}
Az implementáció során létrehozott nyílt forráskódú kód elérhető az alábbi linken: \\ \url{https://github.com/zihbot/tobbvaltozos-diszkretizacio}

\section{Egyszerű példa} \label{appendix:simpleexample}
Az implementáció használata az Iris adatsoron, ismert topológiai sorrend alapján
\begin{lstlisting}
import bediscretizer
import sklearn.datasets

iris = sklearn.datasets.load_iris()
data = bediscretizer.util.concat_array(iris['data'], iris['target'])
d = bediscretizer.MultivariateDiscretizer(data, 'Iris')
d.learn_structure(order=[4,0,1,2,3])
print(d.discretization)
\end{lstlisting}

\label{page:last}
\end{document}
