%----------------------------------------------------------------------------
\chapter{Többváltozós diszkretizáció elméleti háttere}
%----------------------------------------------------------------------------

\section{A probléma meghatározása}
Többváltozós diszkretizáció során arra szeretnénk megoldást találni, hogy egy adatsorban előforuló folytonos változókat intervallumokba soroljunk úgy, hogy az egy intervallumból vett értékek hasonló mértékben korreláljanak egymással és a diszkrét változók értékeivel \cite{monti2013multivariate}.

\section{Jelenlegi módszerek}
\subsection{Minimális hosszúságú leírás elve}
A minimális hosszúságú leírás elve (MDL) azt mondja ki, hogy egy adathalmaz legjobb modellje az, amelyik minimalizálja a leírásához szükséges információ mennyiségét \cite{friedman1996discretizing}.

Az információ sűrűségének jellemzésére megfelelő mutató az (információ) entrópia. Így a feladatot megfogalmazhatjuk úgy is, hogy minimalizálandó a diszkretizációs elv szerint a modell entrópiája.

Ennek megoldása $O(n^3)$ időben futtatható, ahol az $n$ a tanító adatok száma.

\subsection{Bayes döntés}
A diszkretizációs elv kiválasztható úgy, hogy mindegyikre kiszámoljuk \textit{a posteriori} becsléssel, mennyire valószínűek a megadott adatokon, és ezek közül kiválasztjuk a legnagyobbat. A Bayes-tétel miatt ez felírható a likelihood és a prior szorzataként.
$$ \argmax_{\Lambda}{P(\Lambda | D)} =
\argmax_{\Lambda}{P(D | \Lambda) \cdot P(\Lambda)} $$
ahol $\Lambda$ a diszkretizációs elv, $D$ az adat.

Az \textit{a posteriori} becslés esetén a prior megfelelő megválasztása kulcsfontosságú. A könnyebb programozhatóság kedvéért a valószínűség helyett annak negatív logaritmáltját érdemes minimalizálni. Mivel a logaritmus szigorúan monoton növekvő függvény, ez nem változtat az eredményen, de a lebegőpontos számábrázolás miatt pontosabban számolható számítógépen.

Mivel megoldható a feladat dinamikus programozással (lehet építeni a korábban kiszámolt valószínűségekre), így a futási idő $O(n^2)$-re csökken, ahol $n$ a tanító adatok száma.

