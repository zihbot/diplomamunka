%----------------------------------------------------------------------------
\chapter{Bayes-döntés alapján diszkretizáló modul}\label{chapter:modul}
%----------------------------------------------------------------------------
% végére szekvenciadiagrammok, összehasonlítva az előző fejezet algoritmusával

\iffalse
A Diplomamunka részeként elkészült egy \textit{python} modul\footnote{A python nevezéktanban a függvénykönyvtárakat nevezik moduloknak.}, amely a bemutatott algoritmus használatát könnyíti meg. Az eredeti cikkben bemutatott implementáció \textit{julia} nyelven történt, amely egy gyors és dinamikusan fejlődő nyelv, melyet statisztikai feladatok megoldására gyakran használnak. Az új modul a python nyelven íródott, mert jelenleg ez a nyelv elterjedtebb, és az egyik jelentős előnye, hogy nagy mennyiségű függvénykönyvtárral rendelkezik, és ennek a bővítése segítség lehet a többi felhasználónak.

A modul célja, hogy egy robusztus többváltozós diszkretizációt hajtson végre a kapott adatsoron, valamint a kinyerhető információkat a felhasználó rendelkezésére bocsássa. A \verb|bediscretizer| név a "Bayesian Estimation based DISCRETIZER" rövidítése.

\section{Telepítés}
A modul jelenleg nem telepíthető pip-pel. Az \ref{appendix:codebase} függelékben szereplő linkről letölthető a bediscretizer könyvtár. Ezt a python HOME könyvtárának (pl: \verb|C:\Python39|) \verb|Lib| alkönyvtárába bemásolva a modul telepítésre került.

\section{Használat}
Ha a modul telepítve van, a hozzá tartozó python értelmezőben futtatott \verb|import bediscretizer| parancs után felhasználhatóak a biztosított függvények.

A modul központi osztálya a \verb|MultivariateDiscretizer|. Példányosításkor meg kell adni a statisztikai adatsort egy numpy tömbként. Ezen kívül megadható az adatsor neve, amelyet később a kiírásokhoz használ.

Ezután a \verb|fit()| függvényének meghívásával lehet a tanító algoritmust futtatni. Ennrk megadható, hány alkalommal futtassa a diszkretizáció és struktúra tanulási algoritmusokat, ez az epochok száma.
A futtatás után kinyerhetőek az adatok a modellből. Egy egyszerű példa látható a \ref{appendix:simpleexample} függelékben.

\section{Elérhető információk}
A \verb|MultivariateDiscretizer| osztályból az adatsorról sok adat kinyerhető. Ezeknek a listája, és a hozzájuk tartozó leírás szerepel itt.

\subsection{Tagváltozók}
\begin{itemize}
    \item[data] A megadott adatsor az előfeldolgozás után. Egy mátrix csak számokkal feltöltve.
    \item[column\_types] Az automatikusan felismert változótípus, amely lehet \verb|DISCRETE| vagy \verb|CONTINUOUS|.
    \item[discretization] A folytonos változókhoz tartozó diszkretizációs intervallumok határai.
    \item[graph] Az adatsorhoz illesztett gráf.
    \item[name] Az adatsor neve, amelyet az adatsorból kinyert adatok kijelzéséhez használ.
\end{itemize}

\subsection{Metódusok}
\begin{itemize}
    \setlength{\itemindent}{5em}
    \item[fit] A tanító algoritmus futását indítja el.
    \item[show] Az aktív \verb|pyplot figure| gafikonjára kirajzolja az illesztett gráfot.
    \item[draw\_structure\_to\_file] A paraméterként megadott fájlba menti az illesztett gráf képét.
    \item[as\_dataframe] Az adatsort \verb|pandas DataFrame| típusúként adja vissza.
\end{itemize}
\fi

Ez a fejezet bemutatja a Diplomamunka részeként elkészült bediscretizer modul implementációját. Leírja, milyen elvárások, követelmények vannak a csomaggal szemben. Ismerteti a mérnöki döntéseket, amelyeket a csomag létrehozásakor kellett meghozni.

A csomag Python nyelven írodott. Elsődleges célja, hogy egy robusztus többváltozós diszkretizációt hajtson végre a kapott adatsoron. A \verb|bediscretizer| név a "Bayesian Estimation based DISCRETIZER" rövidítése.

\section{Követelmények}
A csomaggal szemben támasztott követelmények a FURPS modell \cite{grady1992practical} szerint vannak csoportosítva.
A tanulás történjen felügyelet nélkül. A végeredménye egy betanított Bayes-háló legyen.