%----------------------------------------------------------------------------
\chapter{Saját implementáció}
%----------------------------------------------------------------------------

A félév során az előző fejezetben bemutatott algoritmust implementáltam \textit{python} nyelven. Ebben a fejezetben bemutatom a felhasznált könyvtárakat, valamint a felhasznált adatsort.

\section{Könyvtárak}
Korábbi ismereteim alapján választottam ki az adatelemzéshez szükséges könyvtárakat

\begin{labeling}{scikit-learn}
\item[numpy] matematikai műveletek és gyors tömbműveletek végzésére
\item[scikit-learn] speciális műveletek (kombináció-számítás) elvégzésére
\item[pandas] adatsorok könnyebb kezelésére
\item[matplotlib] grafikonok készítése
\item[networkx] gráfok kezeléséhez (szülők, gyermekek megtalálása)
\item[pomegranate] Bayes-háló struktúra tanulásához
\end{labeling}

Ezen kívül beépített python könyvtárakat is felhasználtam, valamint a saját implementációmat is felbontottam a struktúra és diszkretizáció tanulására külön, valamint egyéb segédfüggvényekre. A fő fájlt a \textit{Visual Studio Code} Jupyter notebook kiegészítőjével futtattam.

\section{Adatsor}
A SOTE laboratóriumi méréseinek adatsorát használtam az algoritmusok teszteléséhez. Ebben nagy mennyiségű páciens vérkép adatai vannak anonimizáltan.

\subsection{Jellemzők}
A felhasznált jellemzőket P. Fuster-Parra et al. \cite{fuster2016bayesian} szívrendszeri megbetegedési kutatásai nyomán választottam ki. Az általuk létrehozott Bayes-háló a \ref{fig:cardio-bayes} ábrán látható. Ebből az adatsorban a \emph{Gender, Age, HDL, Glucose, TG} szerepelnek rendre \emph{sex, age, HDLC, GLU, TG} néven. A paraméterek feloldása a \ref{tab:parameterek_jelentese} táblázatban szerepel. A HDL helyett a HDLC-t választottam, mindkettő a koleszterinszintet mutatja, de a HDLC-hez több adat kapcsolódott.

\begin{figure}
    \centering
    \includegraphics[width=12cm]{figures/cardio-bayes.png}
    \caption{A szívrendszeri megbetegedés Bayes-hálója}
    \label{fig:cardio-bayes}
\end{figure}

\begin{table}[]
\begin{tabular}{llc}
Paraméter & Jelentés & Referencia tartomány (mmol/L) \\
\hline
sex  & páciens neme &         \\
age  & páciens életkora & 44.0 - 65.0\footnotemark\\
HDLC & koleszterinszint & 0.9 - 1.5     \\
TG   & trigliceridek szintje & 0.5 - 2.0 \\
GLU  & vércukorszint    & 3.9 - 5.8
\end{tabular}
\caption{Paraméterek jelentése}
\label{tab:parameterek_jelentese}
\end{table}
\footnotetext{Az életkor referencia tartománya évben mért, és csak a diszkretizáció miatt szükséges meghatározni}

Az eredetileg szívproblémák feltárására szánt paramétereket választottam, mert ehhez több kutatás kapcsolódik, amelyekben szakértők is segítettek a Bayes-hálók elkészítésében. Így az eredményekhez van összehasonlítási alap.

\subsection{Előfeldolgozás}
Az adatbázisban a mérések külön rekordonként szerepelnek. Így mindhárom laborparaméterre szűrést kellett végeznem. Az így kapott adatsorban egy pácienshez több mérés is társult ugyanarra a paraméterre. Ezekből az elsőt választottam mindenkinél. Ezután megkerestem azokat a pácienseket, akiknél mindhárom paraméterhez tartozott mérés, és az ő azonosítójuk alapján összeillesztettem a résztáblákat. Az illesztés típusa \emph{inner join} művelet volt, így a kapott táblában csak a mindhárom megmért paraméterrel rendelkezőek kerültek.

A külön táblákban található HDLC: 177473, GLU: 269657, TGL: 264209 főből az illesztett táblába csak 119852 került. A szűk keresztmetszet a HDLC volt. Ezen mérések pácienseinek közel kétharmadához tartozott - ez alapján - mérés a másik két paraméterre is.

Az így kapott adathalmaz teljes lett, minden rekord minden mezeje ki lett töltve. Az adathalmaz paramétereinek eloszlása a \ref{fig:eloszlas} ábrán, az együttes eloszlásaik a \ref{fig:egyuttes} ábán látható.

\begin{figure}[htp]
    \centering
    \includegraphics[width=8cm]{figures/age.png}
    \includegraphics[width=8cm]{figures/hdlc.png}
    \includegraphics[width=8cm]{figures/tg.png}
    \includegraphics[width=8cm]{figures/glu.png}
    \caption{Paraméterek eloszlása}
    \label{fig:eloszlas}
\end{figure}

\begin{figure}[htp]
    \centering
    \includegraphics[width=8cm]{figures/age-tg.png}
    \includegraphics[width=8cm]{figures/hdlc-tg.png}
    \includegraphics[width=8cm]{figures/glu-hdl.png}
    \includegraphics[width=8cm]{figures/glu-tg.png}
    \caption{Együttes eloszlások}
    \label{fig:egyuttes}
\end{figure}

\section{Felmerült problémák}
A laboradatokat először futtatva az összes adaton próbáltam tanítani. Ekkor a dinamikus programozás nagy tárhelyigénye miatt 132 GB memóriára lett volna szüksége. Így úgy döntöttem, hogy véletlenszerűen kiválasztott 200 adatponttal dolgozom.

Többször végigfuttattam az algoritmust, eleinte túl sok intervallumot talált. Ennek oka, hogy az L túl magas értéken inicializálódott, így azt levettem konstans 3-ra (laborparaméterekhez jól illeszkedőnek tűnt). Emellett az éppen folytonos adatokat a sorbarendezés után szűrtem az 5\% - 95\% közötti részére, hogy a kiugró értékek ne torzítsák az eredményt.