%----------------------------------------------------------------------------
\chapter{Fogalommagyarázat}
%----------------------------------------------------------------------------

Az alábbi fejezet a Diplomamunkában használt fogalmak jelentését, a megértéshez szükséges alapismereteket foglalja össze.

\section{Előfeldolgozás}
Egy statisztikai feldolgozásra kapott adatsoron nem alkalmazhatóak azonnal statisztikai módszerek. Az adatsorban különböző hibák előfordulhatnak. Ilyenek lehetnek:
\begin{itemize}
    \item Hiányok
    \item Nem megfelelő típusú értékek
    \item Inkozisztens adatok
\end{itemize}
Ezért a statisztikai vizsgálat előtt szükség van egy \emph{előfeldolgozás}i lépésre, amely során előáll az adatok megfelelő formája.

\section{Statisztikai változó, adatpont}
A statisztikai számításnál rendezett adatsorokat használunk. A rendezettség a sorokra és oszlopokra bontott adatokban jelenik meg. Ezért az adathalmaz felírható táblázatos formában. A táblázat oszlopai a statisztikai változók, vagy röviden \emph{változó}k. Ezeket a tulajdonságokat értékeljük ki a vizsgált entitásoknál. Az oszlopban szereplő értékek egyforma típusúak (szám, szöveg), értelmezésük azonos. Amennyiben egy fizikai mennyiséget ábrázol, akkor a mértékegysége minden értéknek azonos, vagy egyértelműen jelölt. Ha a mértékegységek különböznek, az előfeldolgozás során célszerű ugyanolyan mértékegységre váltani őket.

Az adatsor soraiban egy egyedhez tartozó értékek szerepelnek. Ezek jellemzik az adott egyedet az adatsor változói szerint. Az egyed tágabban is értelmezhető, nem szükséges hozzá a valódi világnak egy objektuma, de az egy sorban szereplő adatoknak kapcsolódniuk kell egymáshoz. Egy sort \emph{adatpont}nak is neveznek, mert a változók által kifeszített térben pontosan egy pontot jelöl ki.

A táblázatos forma miatt az adatsor nagy hasonlóságot mutat egy relációs adatbázis táblával vagy nézettel. A fogalmak is összepárosíthatóak. A statisztikai változó megfelel egy adatbázis attribútumnak, mezőnek. A statisztikai adatpont pedig értelmezhető egy adatbázis rekordként. Emiatt a kapcsolat miatt egy relációs adatbázisból kiolvasott adatokkal egyszerűen lehet statisztikai számításokat végezni.

\section{Folytonos és diszkrét változó}
A statisztikai változók legfontosabb csoportosítása \cite{statokosvaltozok}. \emph{Folytonos} változók közé tartoznak azok, melyeknek értéke bármilyen valós számot felvehet, esetleges intervallum megszorítások mellett. Vagyis a felvehető értékek száma végtelen. Az adatsorban természetesen véges mennyiségű adat fog szerepelni, tehát egy folytonos változó \emph{megfigyelt} értékei megszámlálhatóak, de végtelen érték értelmezhető lenne az adott változó értékeként. Vannak egymáshoz közeli, és egymástól távolabbi értékek, a távolságuk kivonással megállípítható. Például ilyen a pontos életkor tizedes törtként években megadva, vagy egy növény levelének a hossza.

Ezzel szemben a \emph{diszkrét} változók értékei végesek, pontosan meghatározható, hogy két adatpont egy diszrét változóban megegyezik vagy eltér. A diszkrét változóknak nincs távolságuk, páronként különbözőek. Ilyen például a biológiai nem, vagy egy növény fajtája.

\section{Regresszió és klasszifikáció}
Gépi tanulási modellek csoportosításának egy módja. Ezek a modellek egy statisztikai adatsorból állnak elő, a vizsgált csoportosítás pedig a kimenetük alapján történik. A modelleknek a kiértékelés során van egy célváltozója, amely értékek a kiértékelendő adatok között nem szerepelnek, és amely értékét a modell alapján szeretnénk meghatározni. Amennyiben ez a célváltozó folytonos, úgy \emph{regresszió}ról beszélünk, ha pedig diszkrét, akkor \emph{klasszifikáció}ról. Több gépi tanulásos modell esetében a klasszifikáció is egyfajta regresszió, mert a modell folytonos értékéket állít elő, azt diszkretizálva kaphatóak meg az osztályok.

\section{Osztály}
A diszkrét változók értékeit \emph{osztály}oknak nevezzük. Az osztályozás feladata meghatározni, hogy az adott diszkrét változó melyik értéket veszi fel. Az osztályozás emiatt megegyezik a klasszifikációval, a magyar szakirodalomban így használják, ám a regresszió párjaként klasszifikációként hivatkoznak rá.

\section{Kvantálás}
A fizikai mérőeszközök analóg módon tudják mérni a jelet. Az adattárolás napjainkban nagy többségében digitális alapú. Digitálás számábrázolásnál nem lehet végtelenül pontosan eltárolni az értékeket, el kell férjen az operációs rendszertől és programozási nyelvtől függő, 32-128 bájtos adatként. Az analóg-digitális konverzió során ezért az analóg eszköz által mért érték kerekítve lesz. Ez nem okoz gondot, hiszen az eszköznek mérési hibája is van, ami általában nagyobb a kerekítési hibánál. Ezt a kerekítési folyamatot nevezik \emph{kvantálás}nak. Ebből adódóan az adatsor folytonos értékei is kisebb számosságú almazból kerülnek ki, mint a valódi értékek, hiszen több érték is ugyanarra a kvantálási szintre lesz kerekítve. Ezért mért értékek esetében érdemes figyelembe venni, hogy még a folytonos változók értékei között is lehetnek megegyezőek.

\section{Diszkretizáció}
A diszkretizáció folyamata során egy folytonos változóból egy diszkrét változó lesz. A folyamat végrehajtása olyan algoritmussal történik, amely a folytonos változónak nem csak az adatsorban szereplő értékeihez tud osztályt rendeli, hanem minden lehetséges értékéhez. Tágabb értelemben azt az algoritmust is \emph{diszkretizáló algoritmus}nak nevezzük, amely előállít egy olyan függvényt, amely segítségével diszkretizáció végezhető. A Diplomamunka ebben az értelemben használja a fogalmat.

\section{Feltételes valószínűségi tábla}
Néhány diszkrét valószínűségi változó közötti összefüggés megadására alkalmas táblázat. Tartalmazza az egyik változó feltételes valószínűségét a többi változóra vonatkoztatva. A nagysága a bemeneti változók számával exponenciálisan változik.

%Kardinalitás
%diszkretizációs intervallum
%hisztogram