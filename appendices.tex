%----------------------------------------------------------------------------
\appendix \label{chapter:függelék}
%----------------------------------------------------------------------------
\chapter*{Függelék}\addcontentsline{toc}{chapter}{Függelék}
\setcounter{chapter}{6}  % a fofejezet-szamlalo az angol ABC 6. betuje (F) lesz
\setcounter{equation}{0} % a fofejezet-szamlalo az angol ABC 6. betuje (F) lesz
\numberwithin{equation}{section}
\numberwithin{figure}{section}
\numberwithin{lstlisting}{section}
%\numberwithin{tabular}{section}


%New colors defined below
\definecolor{codegreen}{rgb}{0,0.6,0}
\definecolor{codegray}{rgb}{0.5,0.5,0.5}
\definecolor{codepurple}{rgb}{0.58,0,0.82}
\definecolor{backcolour}{rgb}{0.95,0.95,0.92}

%Code listing style named "mystyle"
\lstdefinestyle{mystyle}{
  backgroundcolor=\color{backcolour},   commentstyle=\color{codegreen},
  keywordstyle=\color{magenta},
  numberstyle=\tiny\color{codegray},
  stringstyle=\color{codepurple},
  basicstyle=\ttfamily\footnotesize,
  breakatwhitespace=false,         
  breaklines=true,                 
  captionpos=b,                    
  keepspaces=true,                 
  numbers=left,                    
  numbersep=5pt,                  
  showspaces=false,                
  showstringspaces=false,
  showtabs=false,                  
  tabsize=2
}

\section{Kódbázis} \label{appendix:codebase}
Az implementáció során létrehozott nyílt forráskódú kód elérhető az alábbi linken: \\ \url{https://github.com/zihbot/tobbvaltozos-diszkretizacio}

\section{Egyszerű példa} \label{appendix:simpleexample}
\begin{lstlisting}
import bediscretizer
import sklearn.datasets

iris = sklearn.datasets.load_iris()
data = bediscretizer.util.concat_array(iris['data'], iris['target'])
d = bediscretizer.MultivariateDiscretizer(data, 'Iris')
d.fit()
d.draw_structure_to_file()
\end{lstlisting}